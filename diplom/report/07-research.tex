\chapter{Исследовательский раздел}

\section{Подбор параметров сети}
Для оценки разработанного метода вводятся следующие метрики качества:
\begin{itemize}
	\item коэффициент попадания -- отношение числа обращений к страницам, которые уже загружены в буфер, к общему числу обращений;
	\item коэффициент совпадения -- отношение количества совпавших с оптимальным алгоритмов кандидатов на замещение с общим числом запросов поиска страниц для вытеснения.
\end{itemize}

Размер скрытых слоев модели подбирался экспериментально.
Графики зависимости коэффициента совпадения в зависимости от эпохи обучения для различных размеров скрытых слоев на обучающей и тестовой выборках представлены на рисунках \ref{img:test_sizes_train} и \ref{img:test_sizes_test} соответственно.
\includeimage
{test_sizes_train} % Имя файла без расширения (файл должен быть расположен в директории inc/img/)
{f} % Обтекание (без обтекания)
{H} % Положение рисунка (см. figure из пакета float)
{0.8\textwidth} % Ширина рисунка
{Точность модели для различных размеров скрытых слое на тренировочной выборке} % Подпись рисунка

\includeimage
{test_sizes_test} % Имя файла без расширения (файл должен быть расположен в директории inc/img/)
{f} % Обтекание (без обтекания)
{H} % Положение рисунка (см. figure из пакета float)
{0.8\textwidth} % Ширина рисунка
{Точность модели для различных размеров скрытых слое на тестовой выборке} % Подпись рисунка

Исходя из полученных результатов, настраиваемые параметры модели: $d_z$, $d_b$, $d_h$ и $d_v$ были выбраны равными 448, а $d_f$ -- 32.

\section{Сравнение с аналогами}
Сравнение коэффициентов попадания для разработанного метода и существующих аналогов приведено на рисунке \ref{img:hits_comp}.

\includeimage
{hits_comp} % Имя файла без расширения (файл должен быть расположен в директории inc/img/)
{f} % Обтекание (без обтекания)
{H} % Положение рисунка (см. figure из пакета float)
{0.8\textwidth} % Ширина рисунка
{Коэффициент попадания в зависимости от числа обращений для различных методов} % Подпись рисунка

Из графиков видно, что коэффициент попадания для разработанного метода в среднем на 0.02 выше чем для алгоритма clock, который в настоящее время используется в PostgreSQL.
Также коэффициент попадания для разработанного метода на 0.08 ниже, чем у оптимального алгоритма.
Таким образом, разработанный метод лучше существующий аналогов, но все еще имеет возможность для улучшения.

\section{Сравнение различных размеров буфера}

Было проведено сравнение точности модели на тренировочной и тестовой выборках при различных размерах буфера: 64, 128 и 256 страниц.
Результаты для тренировочной и тестовой выборках представлена на рисунках \ref{img:buf_sizes_train_res} и \ref{img:buf_sizes_test_res} соответственно.

\includeimage
{buf_sizes_train_res} % Имя файла без расширения (файл должен быть расположен в директории inc/img/)
{f} % Обтекание (без обтекания)
{H} % Положение рисунка (см. figure из пакета float)
{0.8\textwidth} % Ширина рисунка
{Коэффициент совпадения в зависимости от эпохи обучения для различных размеров буфера на тренировочной выборке} % Подпись рисунка

\includeimage
{buf_sizes_test_res} % Имя файла без расширения (файл должен быть расположен в директории inc/img/)
{f} % Обтекание (без обтекания)
{H} % Положение рисунка (см. figure из пакета float)
{0.8\textwidth} % Ширина рисунка
{Коэффициент совпадения в зависимости от эпохи обучения для различных размеров буфера на тестовой выборке} % Подпись рисунка

Из полученных графиков видно, что чем больше размер буфера, тем ниже точность модели как на тренировочной, так и на тестовой выборках.
Это связанно с тем, что увеличение размера буфера приводит к увеличению числа обучаемых параметров модели и числа возможных вариантов ответов.
Таким образом, метод может оказаться неэффективным при большом размере буфера.
По умолчанию разделяемых кэш буфер Postgres содержит 128 страниц.
Оценка коэффициентов попадания для разработанного метода и аналогов на таком размере буфера показала, что разработанный метод в среднем на два процента лучше аналогов по этому показателю.

\section{Вывод}
Проведенные исследования позволили определить настраиваемые параметры модели, такие как размеры скрытых слоев $d_z$, $d_b$, $d_h$, $d_v = 448$, $d_f = 32$, обеспечивающие баланс между точностью и вычислительной сложностью. 
Разработанный метод продемонстрировал улучшение коэффициента попадания на 0.02 по сравнению с алгоритмом clock, используемым в PostgreSQL, что подтверждает его практическую эффективность. 
Однако отставание на 0.08 от оптимального алгоритма указывает на потенциал для дальнейшей оптимизации.

Анализ влияния размера буфера показал, что увеличение его объема  приводит к снижению точности модели из-за роста числа обучаемых параметров и вариантов вытеснения.
Результаты, полученные для размера буфера по умолчанию, показали улучшение по сравнению с существующими аналогами по введенным метрикам качества.
Это подтверждает целесообразность внедрения метода в реальные системы с аналогичными настройками.

Таким образом, предложенный метод является перспективным решением для управления замещением страниц в разделяемом кэш буфере PostgreSQL, которое может повысить производительность системы управления базами данных за счет меньшего числа операций, взаимодействующих с диском.