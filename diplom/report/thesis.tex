\documentclass{bmstu}

\bibliography{biblio}

\setlist[itemize]{label={---}}
\setlist[enumerate,2]{label=\arabic*), ref=\arabic*)}

\usepackage{enumitem}

% настройка точек в содержании
\makeatletter
% Настройка плотности точек
\renewcommand{\@dotsep}{1.5} 

% Для глав
\renewcommand*\l@chapter[2]{%
	\@dottedtocline{0}{0em}{1.5em}{\bfseries #1}{#2}}

% Для разделов
\renewcommand*\l@section{\@dottedtocline{1}{1.5em}{2.3em}}
\renewcommand*\l@subsection{\@dottedtocline{2}{3.8em}{3.2em}}
\renewcommand*\l@subsubsection{\@dottedtocline{3}{7.0em}{4.1em}}

\titlespacing*{\section}
{0pt}{5.5ex plus 1ex minus .2ex}{4.3ex plus .2ex}
\titlespacing*{\subsection}
{0pt}{5.5ex plus 1ex minus .2ex}{4.3ex plus .2ex}

\begin{document}


\makethesistitle
{Информатика и системы управления} % Название факультета
{Программное обеспечение ЭВМ и информационные технологии} % Название кафедры
{Метод замещения страниц в разделяемом кэш буфере Postgres с использованием нейронных сетей} % Тема работы
{Мицевич~М.~Д./ИУ7-43М} % Номер группы/ФИО студента (если авторов несколько, их необходимо разделить запятой)
{Тассов~К.~Л.} % ФИО научного руководителя
{} % ФИО консультанта (необязательный аргумент; если консультантов несколько, их необходимо разделить запятой)
{Мальцева~Д.~Ю.} % ФИО нормоконтролера

\setcounter{page}{5}

\chapter*{РЕФЕРАТ}
\addcontentsline{toc}{chapter}{РЕФЕРАТ}

%Расчетно-пояснительная записка к выпускной квалификационной работе содержит  \begin{NoHyper}\pageref{LastPage}\end{NoHyper} страницы, \totfig~иллюстраций, 29 источников, 2 приложения.

%В данной выпускной квалификационной работе представлена разработка метода замещения страниц в СУБД Postgres с использованием нейронных сетей.

%Данная выпускная квалификационная работа посвящена исследованию методов замещения страниц, которые могут быть использованы в разделяемом кэш буфере СУБД Postgres.
%Все методы замещения страниц являются приближением оптимального алгоритма, который предполагает исключать страницу, к которой дольше всего не будет обращений в будущем.
%В данной работе описывается метод замещения страниц, который выбирает страницу для замещения с использованием нейронных сетей.

%В работе рассмотрена задача замещения страниц в СУБД Postgres. 
%Рассмотрены различные методы замещения страниц. 
%Проведен сравнительных анализ различных архитектур нейронных сетей, а также подходов к их обучению. 
%Реализован метод замещения страниц с использование нейронных сетей и проведено исследование точности обученной модели в зависимости от различных значений настраиваемых параметров.
%Проведено сравнение коэффициентов точности и попадания разработанного метода с существующими аналогами.



Расчетно-пояснительная записка к выпускной квалификационной работе «Метод замещения страниц в разделяемом кэш буфере Postgres с использованием нейронных сетей» содержит \begin{NoHyper}\pageref{LastPage}\end{NoHyper}страниц, 4 раздела, \totfig~рисунков, 0 таблиц и список используемых источников из 29 наименований.

Ключевые слова: замещение страниц, рекуррентные сети, Postgres.
	
Объект разработки -- метод замещения страниц в разделяемом кэш буфере Postgres.

Цель работы: разработка метода замещения страниц в разделяемом кэш буфере Postgres с использованием нейронных сетей. 

В первой части работы выполнен анализ существующих методов замещения страниц. 
Изучены принципы работы разделяемого кэш буфера PostgreSQL.
Проведен сравнительный анализ нейронных сетей, подходящих для решения задачи.
Сформулирована цель и формализована постановка задачи в виде IDEF0-диаграммы.

Во второй части разработан метод замещения страниц в разделяемом кэш буфере PostgreSQL.
Описаны основные особенности предлагаемого метода.
Сформулированы ограничения предметной области.
Изложены основные этапы разрабатываемого метода в виде детализированной диаграммы IDEF0 и схем алгоритмов.
Выделены функции и структуры СУБД PostgreSQL, которые используются при работе с кэш буфером.

В третьей части обоснован выбор программных средств реализации метода замещения страниц в разделяемом кэш буфере.
Создана обучающая выборка с помощью логирования обращений к буферу. 
Приведены примеры работы программы. 
Описаны используемые методы тестирования программного обеспечения и приведены его результаты.

В четвертой части проведено исследование разработанного метода и выявлена зависимость коэффициентов попадания и совпадения от количества обращений к страницам на тестовой выборке.
Проведено сравнение полученных результатов и значений этих метрик для существующих аналогов.

Разработанный метод замещения страниц может быть использован в СУБД Postgres. 
Использование метода позволит повысить коэффициент попадания в разделяемом кэш буфере, что должно привести к уменьшению времени отклика системы. 

\maketableofcontents

%\include{01-definitions}
%\include{02-abbreviations}

\chapter*{ВВЕДЕНИЕ}
\addcontentsline{toc}{chapter}{ВВЕДЕНИЕ}

С течением времени были разработаны два основных подхода для пре-
одоления перегрузки памяти [1]. Один из них, называемый свопингом, заклю-
чается в полном размещении процесса в памяти при его запуске на некоторое
время, а затем в его сбросе на диск. Бездействующие процессы хранятся на
диске, освобождая оперативную память. Второй подход, называемый вирту-
альной памятью, позволяет программам запускаться, даже если они находятся
в оперативной памяти лишь частично. В этом случае страницы загружаются
в оперативную память и сбрасываются на диск по мере необходимости во
время выполнения программы.
Процесс сброса страниц на диск и возвращение их в оперативную память
является трудоемким, так как требует обращения к ресурсам жесткого диска и
упирается в его пропускную способность. Минимизация числа таких операций
может повысить производительность системы [2].
Алгоритмы замещения страниц являются важной частью управления
памятью в операционных системах. Они определяют, какие страницы памяти
должны быть удалены из физической памяти при нехватке места и какие
страницы должны быть загружены обратно при необходимости. Различные
алгоритмы замещения страниц имеют разные стратегии и критерии для
принятия решений о замещении страниц.
Целью данной работы является сравнительный анализ методов замеще-
ния страниц. Для достижения поставленной цели нужно выполнить следующие
задачи:

Для достижения поставленной цели требуется выполнить следующие задачи:
\begin{itemize}
	\item провести анализ предметной области управления памятью;
	\item сформулировать критерии сравнения методов;
	\item провести сравнительный анализ методов замещения страниц.
\end{itemize}
\chapter{Анализ предметной области}

\section{Виртуализация}
Под виртуализацией понимается предоставление набора вычислительных ресурсов или их логического объединения, абстрагированное от аппаратной реализации и обеспечивающее при этом логическую изоляцию вычислительных процессов, выполняемых на одном физическом ресурсе \cite{алексеев2019гипервизоры}.

Критерии виртуализируемости системы впервые были сформулированы в статье \cite{virtReq}. В данной статье выделяется три основные критерия, которым должен отвечать монитор виртуальных машин:
\begin{itemize}
	\item изоляция -- каждая виртуальная машина должна быть изолирована и иметь доступ только к назначенным ей ресурсам. Она не должна иметь возможности влиять на работу монитора или других виртуальных машин;
	\item эквивалентность -- любая программа, которая выполняется на виртуальной машине, должна демонстрировать поведение, полностью идентичное ее выполнению на реальной системе. Единственные различия могут быть связаны с доступностью ресурсов (например, может быть ограничен объем доступной памяти для виртуальной машины) и длительностью операций (из-за возможности разделения времени выполнения с другими виртуальными машинами);
	\item эффективность -- статистически преобладающее подмножество инструкций виртуального процессора должно исполняться напрямую хозяйским процессором, без вмешательства монитора виртуальных машин.
\end{itemize}

В зависимости от способа реализации виртуализацию подразделяют \cite{admin} на:
\begin{itemize}
	\item полную -- эмулируются все инструкции гостевой операционной системы;
	\item неполную -- нет сокрытия центрального процессора; 
	\item паравиртуализацию -- используются специальные операционные системы, которые знают, что они запускаются как гостевые;
	\item аппаратную виртуализацию -- для виртуализации используются специальные инструкции процессора;
	\item контейнеризация -- процесс запускается в специальном изолированном пространстве и использует то же ядро, что и основная операционная система.
\end{itemize}

\section{Полная виртуализация}

Первые гипервизоры полностью эмулировали базовое оборудование, определяя виртуальные замены всех основных вычислительных ресурсов: жестких дисков, сетевых устройств, прерываний, аппаратных средств материнской платы, BIOS. Такой режим позволяет запускать любые гостевые операционные системы без изменений и называется полной виртуализацией.

Основным недостатком такого типа виртуализации является снижение производительности системы, связанное с тем, что гипервизор должен постоянно выполнять трансляцию между реальным и виртуальным оборудованием системы \cite{admin}.

Основным преимуществом полной виртуализации является независимость между платформой, под которую была разработана гостевая операционная система, и платформой, на которой она будет запускаться.

\section{Неполная виртуализация}

Все центральные процессоры с режимом ядра и пользовательским режимом имеют набор инструкций, ведущих себя по-разному в зависимости от того, в каком режиме они выполняются, в режиме ядра или в пользовательском режиме.
В их число входят инструкции, осуществляющие ввод-вывод, изменяющие настройки блока управления памятью, такие инструкции называются служебными.
Есть также набор инструкций, которые при выполнении в пользовательском режиме вызывают системные прерывания. 
Такие инструкции называются привилегированными.

Машина может быть подвергнута аппаратной виртуализации, только если служебные инструкции являются поднабором привилегированных инструкций \cite{tanenbaumOS}.
Данное ограничение связанно с тем, что гипервизору для корректной работы гостевой операционной системы необходимо перехватывать от нее и обрабатывать все служебные и привилегированные инструкции.

Гипервизоры, которые выполняют неполную виртуализацию, переписывают часть кода на лету, заменяя проблемные инструкции безопасной кодовой последовательностью, эмулирующей исходную инструкцию.
Перезапись полезна для замены инструкций, являвшихся служебными, но не входивших в число привилегированных.
Такая технологий называется двоичной трансляцией \cite{tanenbaumOS}.

Переписывать абсолютно все служебные инструкции нет необходимости.
В частности, пользовательские процессы на гостевой операционной системе могут выполняться без модификации.
Для служебных инструкций, входящих в состав привилегированных используется схема, при которой вызывается системной прерывание и обработчик гипервизора, который его обрабатывает.
Обычно для реализации этого правила в гипервизорах имеется модуль, которые выполняется в пространстве ядра операционной системы, перенаправляющих системные прерывания к своим обработчикам.

Для реализации данного типа виртуализации для процессора x86 многие компании использовали следующий подход: гипервизор запускался на нулевом кольце защиты, все пользовательские приложения -- на третьем, а гостевая операционная система -- на первом. 
Таким образом, попытка доступа к памяти ядра гостевой операционной системы из пользовательского приложения приведет к нарушению прав доступа, а исполнение привилегированных инструкций гостевой операционной системы приведет к вызову системного прерывания с передачей управления гипервизору.
Для обработки служебных инструкций в коде операционной системе используется динамическая двоичная трансляция, которая заменяет их на вызовы процедур гипервизора.

\section{Паравиртуализация}
Паравиртуализация -- это техника виртуализации, при которой гостевая операционная система подготавливается к исполнению в виртуализированной среде, для чего ее ядро незначительно модифицируется \cite{рыбаков2021виртуализация}.
При паравиртуализации привилегированные инструкции, исполнение которых запрещено в гостевом режиме, подменяются на гипервызовы -- прямые передачи управления гипервизору.
Достойной производительности гостевых систем позволяет добиться характерная
паравиртуализации оптимизация: замена одним гипервызовом нескольких последовательных привилегированных инструкций. 
Однако применение паравиртуализации ограничивается возможностью модификации кодов операционной системы.

Определенные паравиртуализованные функции могут быть реализованы и в гипервизорах с полной виртуализацией с использование специальных виртуальных драйверов в гостевых операционных системах, которые общающаются напрямую с гипервизором.

Главным преимуществом такого подхода является более высокая скорость работы по сравнению с полной и неполной виртуализацией, так для работы гостевой операционной системы не требуется полная эмуляция всех инструкций и динамическая трансляция кода операционной системы.

Главным недостатком такого подхода является необходимость модификации кода гостевой операционной системы.

\section{Аппаратная виртуализация}

В 2004 и 2005 годах компаниями Intel и AMD были представлены функции процессора, которые способствуют виртуализации на платформе x86. 
В этой схеме центральный процессор и контроллер памяти виртуализируются аппаратно, хотя и под управлением гипервизора.
Виртуализация, при которой используются специальные инструкции процессора называется аппаратной \cite{гоношенко2020аппаратно}.
Производительность при этом выше, чем при неполной виртуализации, а гостевые операционные системы не должны знать, что они виртуализированы.

Основной замысел технологии заключается в том, что гостевая операционная система запускается в контейнере и работает в нем до тех пор, пока ее не будет вызвано исключение или не будет осуществлено системное прерывание в гипервизоре.
Набор операций, вызывающих системное прерывание задается гипервизором.

\section{Типы гипервизоров}
Выделяют два основных типа гипервизоров:
\begin{itemize}
	\item первый тип -- гипервизор запускается без основной операционной системы и сам выполняет ее функции;
	\item второй тип -- гипервизор запускается внутри основной операционной системы.
\end{itemize}

В первом случае гипервизор -- единственная программа, которая запущена в привилегированном режиме.

Гипервизоры первого типа используются предприятиями, а второго удобны пользователям для быстрого развертывания тестового стенда.

\section{Контейнерная виртуализация}

Контейнеризация -- это другой подход к изоляции который не использует гипервизор.
Вместо этого он опирается на возможности ядра, изолирующие процессы от остальной системы.
Каждый процесс имеет собственную корневую файловую систему и пространство имен процессов. 
Содержащиеся в нем процессы совместно используют ядро и другие сервисы основной операционной системы, но они не могут получить доступ к файлам или ресурсам за пределами своих контейнеров.
Поскольку данная технология не требует виртуализации аппаратного обеспечения, ресурсные накладные расходы при виртуализации на уровне операционной системы меньше по сравнению с другими подходами.

Контейнеры представляют собой объединение множества существующих функций ядра, возможностей файловой системы и сетевых абстракций.
Контейнерный движок -- это управляющее программное обеспечение, которое управляет работой контейнеров.

Для изоляции процессов контейнерных движок использует следующие функции ядра \cite{jain2020linux}:
\begin{itemize}
	\item пространства имен -- задают, какие ресурсы ядра видны процессу;
	\item контрольные группы -- выделяет и ограничивает ресурсы, такие как процессор, память, сетевой ввод‑вывод, которые используются контейнерами;
	\item многослойные файловые системы -- состоят из нескольких слоев.
\end{itemize}

Образы контейнеров представляют собой многослойную файловую систему,
которая по своей организации напоминают корневую файловую систему дистрибутива Linux.

При создании контейнера в файловую систему образа добавляется еще один слой доступный как для чтения, так и для записи.
При попытке изменения контейнером какого-либо файла в слое доступном только для чтения срабатывает стратегия copy on write и этот файл переносится в слой для записи, созданный при запуске контейнера.

\chapter{Сравнение существующих решений}

\section{Сравнение контейнеров и виртуальных машин}

Обе эти технологии создают иллюзию того, что на одной машине можно запускать несколько машин. 
Все эти машины, работающие под управлением основной машины, должны быть изолированы друг от друга, а также от основной машины.
Разница заключается в том, как обе эти технологии способны достичь изоляции между различными машинами. 
Разница заключается в том, что контейнеры обычно выполняются на хостовой операционной системе, а виртуальные машины -- на гипервизоре.
Контейнерный движок обычно совмещен с ядром хостовой операционной системы.

Виртуальные машины и контейнеры отличаются тем, чем может быть гостевая операционная система.
Виртуальные машины позволяют запускать гостевое ядро, отличное от ядра основной операционной системы.
Это невозможно в случае контейнеров, поскольку ядро должно быть общим.

Контейнеры по сравнению с виртуальными машинами требуют меньше ресурсов \cite{yadav2019docker}. Благодаря этому время запуска контейнеров меньше, чем у виртуальных машин. Репликация также проще в контейнерах, поскольку они не требуют отдельной операционной системы.

Безопасность контейнеров ниже, чем у виртуальных машин, так как все контейнеры разделяют одно ядро, в то время как виртуальные машины работают под управлением отдельной операционной системы, благодаря чему они могут использовать свои собственные функции безопасности и ядра.

\section{Сравнение методов виртуализации}
Главным преимуществом полной виртуализации является возможность запуска гостевой операционной системы разработанной под платформу отличную от той, где запущен гипервизор.
Главным недостатком такого подхода являются накладные расходы на эмуляцию каждой инструкции.
Полная виртуализация может быть полезна, к примеру, для запуска приложений, разработанных под операционную систему Windows и процессор x86, на операционной системе Эльбрус, запущенной на одноименном процессоре.

При неполной виртуализации повышается производительность системы, так как эмулируются не все инструкции, а только служебные и привилегированные.
При таком подходе необходимо, чтобы платформа, под которую разработана гостевая операционная система, совпадала с той, на которой запущен гипервизор.

Паравиртуализация позволяет запускать специальные гостевые операционные системы, которые знают, что они вирутализированы, и используют специальные гипервызовы для обращения к гипервизору.
Накладные расходы на виртуализацию при таком подходе меньше чем при неполной виртуализации, но требуется модификация гостевых операционных систем.

При аппаратной виртуализации гипервизор для создания и управления виртуальными машинами использует специальные инструкции процессора.
Производительность системы при таком подходе выше чем при неполной виртуализации, а для запуска виртуальной машины не требуется модификация гостевой операционной системы.

При контейнеризации отдельные процессы запускаются в изолированном пространстве и разделяют одно ядро основной операционной системы.
При таком подходе снижаются накладные расходы на виртуализацию, и время на запуск.
Недостатком контейнеризации по сравнению с виртуальными машинами является меньшей уровень безопасности и невозможность запуска контейнеров с ядром, отличающимся от основной операционной системы.

Сравнение методов виртуализации представлено в таблице \ref{cmp_tbl}.

\begin{table}[!h]
	\small
	\begin{center}
		\captionsetup{justification=raggedleft,singlelinecheck=off}
		\caption{\label{cmp_tbl}Сравнение методов виртуализации} 
\rotatebox{90}{
	\begin{tabular}{|l|c|c|c|c|c|}
	\hline
	& \multicolumn{1}{l|}{\textbf{Полная}} & \multicolumn{1}{l|}{\textbf{Неполная}} & \multicolumn{1}{l|}{\textbf{Паравиртуализация}} & \multicolumn{1}{l|}{\textbf{Аппаратная}} & \multicolumn{1}{l|}{\textbf{Контейнерная}} \\ \hline
	\textbf{\begin{tabular}[c]{@{}l@{}}Возможность запуска\\ на другой архитектуре\end{tabular}}      & +                                    & -                                      & -                                               & -                                        & -                                          \\ \hline
	\textbf{Любая гостевая ОС}                                                                        & +                                    & +                                      & -                                               & +                                        & -                                          \\ \hline
	\textbf{\begin{tabular}[c]{@{}l@{}}Создание без основной\\ ОС\end{tabular}}                       & +                                    & +                                      & +                                               & +                                        & -                                          \\ \hline
	\textbf{\begin{tabular}[c]{@{}l@{}}Создание без специальных\\ инструкций процессора\end{tabular}} & +                                    & +                                      & +                                               & -                                        & +                                          \\ \hline
	\textbf{\begin{tabular}[c]{@{}l@{}}Запуск как обычного\\ пользовательского процесса\end{tabular}} & -                                    & -                                      & -                                               & -                                        & +                                          \\ \hline
\end{tabular}
}
	\end{center}
\end{table}

\chapter{Конструкторский раздел}

\section{Входные данные}
На вход методу подается атрибуты страницы, к которой происходит обращение, и атрибуты всех страниц, которые уже находятся в буфере.

Для сохранения истории обращений к кэш буферу в функцию ReadBufferExtended был добавлен вызов функции печати в лог файл атрибутов страницы, к которой идет обращение.
Эта функция вызывается, каждый раз, когда необходимо прочитать страницу из буфера.

Для каждой страницы извлекается следующий набор атрибутов:
\begin{itemize}
	\item идентификатор отношения;
	\item номер страницы в файле;
	\item наличие индекса;
	\item позиция в буфере.
\end{itemize}

Если в момент обращения к странице она не находится в буфере, то позиция задается размером буфера.

\section{Проектирование метода}
IDEF-0 диаграмма метода замещения страниц уровня A1 приведена на рисунке \ref{img:idef0A1}.
\includeimage
{idef0A1} % Имя файла без расширения (файл должен быть расположен в директории inc/img/)
{f} % Обтекание (без обтекания)
{H} % Положение рисунка (см. figure из пакета float)
{\textwidth} % Ширина рисунка
{IDEF-0 диаграмма уровня A1} % Подпись рисунка

\textbf{Кодировщик запроса обращения к странице} отвечает за скрытое представление атрибутов страницы, к которой происходит очередное обращение.
Схема кодировщика запроса обращения к странице изображена на рисунке \ref{img:page_acc}.
\includeimage
{page_acc} % Имя файла без расширения (файл должен быть расположен в директории inc/img/)
{f} % Обтекание (без обтекания)
{H} % Положение рисунка (см. figure из пакета float)
{0.6\textwidth} % Ширина рисунка
{Схема кодировщика запроса обращения к странице} % Подпись рисунка

На вход кодировщика поступают $n$ атрибутов страницы.
Каждый атрибут может иметь $m_i$ возможных значений, где $i$ -- индекс атрибута.
Каждый атрибут представляется в виде вектора $a^{(i)}$ размерности $m_i$.
Для категориальных данных используется техника однозначного кодирования, а для числовых -- применяется хэш функция и к полученному результату применяется техника однозначного кодирования.
$W_1^{(i)}$ -- матрица обучаемых весов для скрытого представления i-го атрибута.
Вектор $z$ -- выходной вектор из сети.
$W_1$ -- матрица обучаемых весов, при помощи который получается результирующий вектор из скрытых представлений атрибутов сети.
В качестве функции активации на последнем слое используется функция Relu.
$d_f$ и $d_z$ являются настраиваемыми параметрами, которые отвечают за число нейронов, отвечающий за скрытое представление каждого атрибута, и число нейронов на выходном слое соответственно.

Работу сети можно описать с помощью выражений
\ref{formula:page_enc_1} - \ref{formula:page_enc_3}:
\begin{equation}\label{formula:page_enc_1}
	f^{(i)} = a^{(i)}W_1^{(i)} i \in \{1;n\},
\end{equation}

\begin{equation}\label{formula:page_enc_2}
	f = [f^{(1)}, f^{(2)}, ..., f^{(n)}],
\end{equation}

\begin{equation}\label{formula:page_enc_3}
	z = ReLU(W_1f^T + l_1),
\end{equation}
где $f$ является конкатенацией векторов скрытых состояний атрибутов страницы, а $l_1$ -- обучаемым вектором.

\textbf{Кодировщик страниц в буфере} нужен для скрытого представления каждой страницы в буфере.
Схема кодировщика представлена на рисунке \ref{img:buf_page_enc}
\includeimage
{buf_page_enc} % Имя файла без расширения (файл должен быть расположен в директории inc/img/)
{f} % Обтекание (без обтекания)
{H} % Положение рисунка (см. figure из пакета float)
{0.8\textwidth} % Ширина рисунка
{Схема кодировщика страниц в буфере} % Подпись рисунка

На вход кодировщика поступают $M$ страниц из буфера.
Каждая страница представляется в виде $n$ атрибутов.
Процесс обработки атрибутов для каждой страницы такой же, как и в кодировщике запроса обращения к странице.
Для каждой i-ой страницы в буфере вычисляется вектор $b_i$.
$d_b$ является настраиваемым параметром, который отвечает за размерность векторов $b_i$.

Для получения скрытого представления каждого атрибута и для вычисления закодированного представления страницы используется одни и те же матрицы весов $W_2^{(i)}$ и $W_2$ для всех страниц в буфере.
За счет этого матрицы весов не привязаны к конкретной позиции страницы в буфере и истории страниц на этой позиции.
При обратном распространении ошибки влияние веса из матрицы $W_2$ будет учитываться для всех векторов $b_i$.

Обозначим результат работы сумматора нейрона на выходном слое как $s_{ij}$.
Индексация в матрице $s$ совпадает с матрицей $b$.
Тогда для вычисления ошибки по весу $w_{ij}$ из матрицы $W_2$ на ребре, которое соединяет j-ый нейрон из второго слоя и i-ый нейрон из выходного слоя, используется выражение \ref{formula:buf_page_enc_err}:
\begin{equation}\label{formula:buf_page_enc_err}
	\frac{\delta E}{\delta w_{ij}} = \sum\limits_{k=1}^{M}\frac{\delta E}{\delta b_{ki}}\frac{\delta b_{ki}}{\delta s_{ki}} \frac{\delta s_{ki}}{\delta w_ij},
\end{equation}
где $E$ -- функция ошибки, $\frac{\delta E}{\delta b_{ki}}$ -- ошибка полученная со следующего слоя.

Функционирование кодировщика определяется выражениями
\ref{formula:buf_page_enc_1} - \ref{formula:buf_page_enc_3}:
\begin{equation}\label{formula:buf_page_enc_1}
	f^{(j,i)} = a^{(j,i)}W_2^{(i)} j \in \{1;M\} i \in \{1;n\},
\end{equation}

\begin{equation}\label{formula:buf_page_enc_2}
	f^{(j)} = [f^{(j,1)}, f^{(j,2)}, ..., f^{(j,n)}],
\end{equation}

\begin{equation}\label{formula:buf_page_enc_3}
	b_j = ReLU(W_2f^{(j)T} + l_2),
\end{equation}
где $f^{(j)}$ -- конкатенация скрытых представлений атрибутов для j-ой страницы в буфере, $b_j$ -- скрытое представление этой страницы, $l_2$ -- вектор обучаемых весов.

\textbf{Кодировщик истории обращений.} Для обновления истории обращений используется сеть LSTM.
На вход сети поступают результат работы кодировщика обращения к странице, предыдущий результат кодировщика истории обращений и предыдущее состояние ячейки.

Функционирование кодировщика определяется выражениями
\ref{formula:lstm_enc_1} - \ref{formula:lstm_enc_6}:
\begin{equation}\label{formula:lstm_enc_1}
	f_t = \sigma(W_f[h_{t-1}, z_t] + b_f),
\end{equation}

\begin{equation}\label{formula:lstm_enc_2}
	i_t = \sigma(W_i[h_{t-1}, z_t] + b_i),
\end{equation}

\begin{equation}\label{formula:lstm_enc_3}
	\hat{C_t} = \tanh(W_C[h_{t-1}, z_t] + b_C),
\end{equation}

\begin{equation}\label{formula:lstm_enc_4}
	C_t = f_t * C_{t-1} + i_t*\hat{C_t},
\end{equation}

\begin{equation}\label{formula:lstm_enc_5}
	o_t = \sigma(W_o[h_{t-1}, z_t] + b_o),
\end{equation}

\begin{equation}\label{formula:lstm_enc_6}
	h_t = o_t * \tanh(C_t),
\end{equation}

где $[h_{t-1}, z_t]$ -- конкатенация результата работы предыдущего слоя кодировщика истории и скрытого состояния, полученного из кодировщика обращения к странице, $W_f$ и $b_f$ -- матрица и вектор обучаемых весов, $f_t$ -- результат работы фильтра забывания, $i_t$ определяет, какие значения будут сохранены в ячейке, $\hat{C_t}$ -- новые значения кандидатов на попадание в ячейку, $W_i$, $W_C$, $b_i$, $b_c$ -- матрицы и вектора обучаемых весов, $C_t$ -- новое состояние ячейки, $C_{t-1}$ -- состояние ячейки на прошлом шаге, $h_t$ -- результат работы текущего слоя, $C_t$ -- состояние ячейки, $W_o$ и $b_o$ -- матрица и вектор обучаемых весов.
Вектора $h_t$ и $C_t$ имеют размерность $d_h$, где $d_h$ -- настраиваемый параметр.

\textbf{Модуль выбора страниц для замещения.}
На вход модуля поступают результаты работы кодировщика страниц в буфере и кодировщика истории обращений.
Для выбора страницы, которая будет удалена из буфера используется указательная нейронная сеть с механизмом внимания \cite{vinyals2015pointer}.

Нейронные сети с механизмом внимания -- это архитектуры, которые позволяют моделям динамически фокусироваться на наиболее релевантных частях входных данных при обработке информации.
Этот подход нашел применения в областях обработки естественного языка, компьютерного зрения и других задач, где важно учитывать контекст и зависимости между элементами последовательности.
В модуле выбора страниц для замещения контекстом является результат работы кодировщика истории, а элементами последовательности -- результаты работы кодировщика страниц в буфере.

Указательные сети -- архитектура сетей с механизмом внимания, предназначенная для решения задач, где выходные элементы представляют собой позиции в входной последовательности.
В указательных сетях механизм внимания используется как указатель на один из элементов входной последовательности, а не для создания контекстного вектора, как в классических моделях с механизмом внимания.

Схема модуля выбора страниц для замещения представлена на рисунке \ref{img:decision_maker}
\includeimage
{decision_maker} % Имя файла без расширения (файл должен быть расположен в директории inc/img/)
{f} % Обтекание (без обтекания)
{H} % Положение рисунка (см. figure из пакета float)
{0.8\textwidth} % Ширина рисунка
{Схема модуля выбора страниц для замещения} % Подпись рисунка

$W_4$ -- матрица обучаемых весов, которая используется для преобразования вектора $h$, полученного из кодировщика истории в вектор контекста размерности $d_v$. $d_v$ является настраиваемым параметром модели.

$W_3$ -- матрица обучаемых весов, которая используется для преобразования закодированного состояния очередной страницы в буфере в вектор размерности $d_v$, который будет использован в функции внимания.

$v$ -- вектор обучаемых весов, который который используется при вычислении функции внимания.

Функционирование модуля выбора страниц для замещения определяется выражениями \ref{formula:evict_1} - \ref{formula:evict_3}:
\begin{equation}\label{formula:evict_1}
	u_i = v * \tanh(W_3 b_i^T + W_4 h^T), i \in \{1; M\},
\end{equation}

\begin{equation}\label{formula:evict_2}
	p_i = SoftMax(u_i),
\end{equation}

\begin{equation}\label{formula:evict_3}
	r = \arg\max_{i} p_i,
\end{equation}
где $M$ -- число страниц в буфере, $u_i$ -- результат функции внимания для i-ой страницы в буфере, $\arg\max_{i} p_i$ -- функция, которая возвращает индекс максимального элемента в последовательности, $r$ -- результат работы спроектированного метода замещения страниц.

Схемы алгоритмов обучения нейронных сетей и прохождения одной эпохи приведены на рисунках \ref{img:training} и \ref{img:training_epoch} соответственно.
\includeimage
{training} % Имя файла без расширения (файл должен быть расположен в директории inc/img/)
{f} % Обтекание (без обтекания)
{H} % Положение рисунка (см. figure из пакета float)
{0.6\textwidth} % Ширина рисунка
{Схема алгоритма обучения нейронной сети} % Подпись рисунка

\includeimage
{training_epoch} % Имя файла без расширения (файл должен быть расположен в директории inc/img/)
{f} % Обтекание (без обтекания)
{H} % Положение рисунка (см. figure из пакета float)
{0.6\textwidth} % Ширина рисунка
{Схема алгоритма прохождения одной эпохи} % Подпись рисунка

\section{Структура программного обеспечения}

Программное обеспечение состоит из пяти модулей:
\begin{itemize}
	\item модуль получения обучающей выборки;
	\item кодировщик запросов обращения к страницам;
	\item кодировщик истории запросов обращения к страницам;
	\item кодировщик страниц в буфере;
	\item модуль выбора страниц для замещения.
\end{itemize}

Структурная схема взаимодействия модулей разрабатываемого программного обеспечения представлена на рисунке \ref{img:modules}.

\includeimage
{modules} % Имя файла без расширения (файл должен быть расположен в директории inc/img/)
{f} % Обтекание (без обтекания)
{H} % Положение рисунка (см. figure из пакета float)
{0.8\textwidth} % Ширина рисунка
{Структура программного обеспечения} % Подпись рисунка

Модуль получения обучающей выборки нужен для обработки лог файла и создания обучающей и тестовой выборок.
Остальные модули предназначены для обучения и взаимодействия с уже обученными моделями.

\section{Набор обучающих данных}
Для имитации нагрузки на СУБД и получения истории обращений к страницам был использован тестовый сценарий TPC-C \cite{leutenegger1993modeling}.

TPC-C -- это стандартный тест для оценки производительности систем, обрабатывающих транзакции в режиме реального времени. 
Он имитирует работу оптового поставщика с распределённой сетью складов и предназначен для тестирования СУБД на реалистичных сценариях высокой нагрузки.

Для обработки нагрузки создается база данных, состоящая из следующих таблиц:
\begin{itemize}
	\item Warehouse (склады);
	\item District (регионы складов);
	\item Customer (клиенты);
	\item Order (заказы);
	\item Order-Line (позиции заказов);
	\item Item (товары);
	\item Stock (запасы на складах);
	\item History (история платежей).
\end{itemize}

TPC-C включает 5 типов транзакций, имитирующих реальные бизнес-операции:
\begin{enumerate}
	\item NewOrder (45\%): создание нового заказа (вставка данных в таблицы Order, Order-Line, обновление Stock).
	\item Payment (43\%): Обработка платежа (обновление Customer, Warehouse, District, вставка в History).
	\item Delivery (4\%): Доставка заказа (удаление из Order, обновление Customer).
	\item OrderStatus (4\%): Проверка статуса заказа (выборка из Order, Order-Line, Customer).
	\item StockLevel (4\%): Проверка уровня запасов (выборка из Stock).
\end{enumerate}

С помощью тестовой нагрузки было получено 6 554 959 обращений к страницам.
Выборка была поделена на тренировочную и тестовую в отношении семь к трем.

Корреляция между атрибутами страниц, посчитанная на всей выборке, представлена на рисунке \ref{img:correlation}.

\includeimage
{correlation} % Имя файла без расширения (файл должен быть расположен в директории inc/img/)
{f} % Обтекание (без обтекания)
{H} % Положение рисунка (см. figure из пакета float)
{0.8\textwidth} % Ширина рисунка
{Корреляция между атрибутами страниц} % Подпись рисунка

Корреляция вычислялась по формуле Пирсона \ref{formula:pirson}:
\begin{equation}\label{formula:pirson}
	c = \frac{\sum\limits_{i=1}^{n} (x_i - \overline{x}) (y_i - \overline{y})}{\sqrt{\sum\limits_{i=1}^{n}(x_i - \overline{x})^2} \sqrt{\sum\limits_{i=1}^{n}(y_i - \overline{y})^2}},
\end{equation}
где $n$ -- число элементов в последовательности, $x_i$ и $y_i$ -- элементы последовательностей, между которыми считается корреляция, $\overline{x}$ и $\overline{y}$ -- средние значения элементов последовательностей.

Количество замещений страницы оптимальным алгоритмом в зависимости от индекса страницы в буфере для тренировочной и тестовой выборок приведено на рисунках \ref{img:evict_freq_train} и \ref{img:evict_freq_test} соответственно.

\includeimage
{evict_freq_train} % Имя файла без расширения (файл должен быть расположен в директории inc/img/)
{f} % Обтекание (без обтекания)
{H} % Положение рисунка (см. figure из пакета float)
{0.8\textwidth} % Ширина рисунка
{Количество замещений страницы по индексу на тренировочной выборке} % Подпись рисунка

\includeimage
{evict_freq_test} % Имя файла без расширения (файл должен быть расположен в директории inc/img/)
{f} % Обтекание (без обтекания)
{H} % Положение рисунка (см. figure из пакета float)
{0.8\textwidth} % Ширина рисунка
{Количество замещений страницы по индексу на тестовой выборке} % Подпись рисунка


\section{Вывод}

В данном разделе были определены ограничения, которые накладываются на входные данные, и требования, которые предъявляются к разрабатываемому программному обеспечению.

Была детализирована IDEF-0 диаграмма уровня А0, описанная в разделе формализованной постановки задачи, а также было проведено разбиение программного обеспечения на модули.
Задача выбора страниц для замещения из буфера была разбита на четыре подзадачи: кодирование запроса обращения к странице, обновление истории обращений, кодирование страниц в буфере, выбор страниц для замещения на основе контекста.

Для решения каждой подзадачи была спроектирована архитектура нейронной сети.
Для обеспечения временных зависимостей при обновлении истории была выбрана сеть LSTM.
Для выбора страницы для замещения была спроектирована указательная сеть с механизмом внимания, которая выбирает одну из страниц в буфере для замещения.
Была определена схема алгоритма обучения составленных нейронных сетей.

Для генерации реалистичной нагрузки использован тестовый сценарий TPC-C, что позволило получить шесть с половинной миллионов обращений к страницам.
Распределение замещений страниц на тренировочной и тестовой выборках подтвердило сбалансированность данных.
\chapter{Технологический раздел}

\section{Средства реализации программного обеспечения}

В качестве языка программирования был выбран Python. Данный выбор обусловлен тем, что Python имеет множество библиотек, таких как TensorFlow, Keras, PyTorch и Theano, которые предоставляют множество инструментов для создания и обучения нейронных сетей.

В качестве библиотеки для создания нейронной сети была выбрана библиотека PyTorch версии 2.0.0, так как она имеет следующие возможности и инструменты:
\begin{itemize}
	\item динамический граф вычислений, использование которого облегчает отладку моделей;
	\item возможность переноса вычислений на GPU;
	\item набор инструментов для создания различных слоев, из которых складывается архитектура нейронной сети;
	\item API на языке C++, что позволяет обучить модель с использованием интерпретируемого языка Python, а использовать ее на компилируемом языке C++.
\end{itemize}

Для работы с большими данными была выбрана библиотека numpy версии 1.21.0, так как она использует оптимизированный код на C, что позволяет выполнять вычисления быстрее, чем с использованием чистого Python, а также потому что классы этой библиотеки интегрируются с библиотекой PyTorch, которая используется для создания нейронной сети.

Для работы с изображениями была выбрана библиотека Pillow версии 8.2.0, так как она позволяет загружать и трансформировать изображения разных форматов, таких как PNG, JPEG, BMP, а также переводить изображения в объекты, которые передаются на вход нейронной сети.

Для создания графического интерфейса использовалась библиотека PyQt версии 5.0, так как она является кроссплатформенной, имеет базовые виджеты, на которых может быть построен интерфейс, а также предоставляет возможность подписки на события, которые генерируются виджетами.
%TODO add refs

\section{Реализация программного комплекса}

В качестве модели классификации используется сверточная нейронная сеть с 12 слоями свертки и 5 слоями пуллинга, которая соединяется с двухслойным перцептроном. После каждого сверточного слоя за исключением последнего используется слой нормализации и функция активации ReLU. Реализация модели приведена в листинге \ref{lst:simple_net.py} (приложение А).

Перед обучением модели выполняется предобработка всего обучающего набора данных, во время которой нужно разбить все входные изображения по классам, привести все изображения к размеру 96 на 96 пикселей, так как только такое изображение обрабатывается моделью, а также поделить весь набор данных на обучающую и тестовую выборки.

После выполнения предобработки нужно произвести аугментацию для расширения обучающей выборки. К каждому изображению в обучающем множестве применяются следующие преобразования: поворот на 30 и 330 градусов, увеличение яркости в полтора раза, а также гауссово размытие, которое создает фильтр Гаусса с заданным размером и силой размытия, и применяют его к изображению, используя операцию свертки. Коэффициенты в фильтре Гаусса вычисляются по формуле \ref{formula:gaussfilter}
\begin{equation}\label{formula:gaussfilter}
G(x,y) = \frac{1}{2\pi\sigma^2}e^{-\frac{x^2+y^2}{2\sigma^2}},
\end{equation}
где $x$ и $y$ -- расстояния от центра ядра по горизонтали и вертикали соответственно, $\sigma$ -- сила размытия.

Исходное изображения и изображения после аугментации приведены на рисунках \ref{img:tensor}-\ref{img:augTensor4}.

Обучение модели проводилось на машине с процессором Intel Core i9-10900, 64 гигабайтами оперативной памяти и графической картой NVIDIA GeForce RTX 3080 с 16 гигабайтами памяти типа GDDR6. Время прохождения одной эпохи для модели классификации в среднем занимает 5 минут 4 секунды, а для модели детектирования -- 6 минут 15 секунд.

В качестве оптимизатора функции потерь был выбран Adam, так как он автоматически адаптирует скорость обучения для каждого параметра в зависимости от его градиента, что позволяет более эффективно использовать скорость обучения и ускоряет сходимость.

Обучение модели останавливается, когда точность распознавания на тестовой выборке после прохождения очередной эпохи становится меньше, чем на двух предыдущих, так как это свидетельствует о том, что сеть начала переобучаться.

Код обработки и загрузки обучающего набора данных представлен в листинге \ref{lst:dataset_handler.py}. Реализация алгоритмов обучения и прохождения одной эпохи представлены на листингах \ref{lst:INetworkController.py} и \ref{lst:NetworkController.py} соответственно. Все листинги находятся в приложении А.

Графики зависимостей точности модели классификации на тестовой и обучающей выборках от номера эпохи обучения приведены на рисунке \ref{img:model_training}.

\begin{figure}[H]
	\begin{center}
		\begin{tikzpicture}
		\begin{axis}[
		xlabel = {номер эпохи},
		ylabel = {точность},
		legend pos=south east,
		ymajorgrids=true,
		width=12cm
		]   
		\addplot[color=blue, mark=square] table[x index=0, y index=1] {inc/data/model_training.dat};
		
		\addplot[color=red, mark=*] table[x index=0, y index=1] {inc/data/model_test.dat};
		
		
		\addlegendentry{Тренировочная}
		\addlegendentry{Тестовая}
		
		\end{axis}
		\end{tikzpicture}
		\caption{Обучение модели}
		\label{img:model_training}
	\end{center}
\end{figure}

Из графиков видно, что после 9 эпохи модель начинает переобучаться и требуется остановить обучение. Итоговая точность полученной модели на тестовой выборке составила 84 процента.

В качестве модели детектирования была реализована сверточная нейронная сеть, состоящая из 106 слоев. В качестве функции активации была выбрана функция ReLU. Разработанная модель имеет три выходных слоя, каждый из которых имеет размерность (3xSxSx5), где 3 -- число якорей, S -- размер ячейки, для 3 выходов размеры ячеей будут 25, 50 и 100 соответственно, вектор из пяти элементов состоит из вероятности нахождения центра объекта внутри ячейки, а также координат его центра, высоты и ширины рамки.

Графики зависимости точности детектирования объектов и вероятности ложного срабатывания представлены на рисунке \ref{img:yolo_training}. Объект считался корректно распознанным, если отношение площади пересечения полученной рамки с эталоном к площади их объединения составляло больше 0.6. Детектированный объект считался ложным, если отношение площади пересечения его рамки ни с одной из эталонных к площади их объединения не составляло больше 0.1.

\begin{figure}[H]
	\begin{center}
		\begin{tikzpicture}
		\begin{axis}[
		xlabel = {номер эпохи},
		ylabel = {точность},
		legend pos=south east,
		ymajorgrids=true,
		width=12cm
		]   
		\addplot[color=blue, mark=square] table[x index=0, y index=1] {inc/data/yolo_error.dat};
		
		\addplot[color=red, mark=*] table[x index=0, y index=1] {inc/data/yolo_test.dat};
		
		
		\addlegendentry{Ложное срабатывание}
		\addlegendentry{Тестовая}
		
		\end{axis}
		\end{tikzpicture}
		\caption{Обучение модели}
		\label{img:yolo_training}
	\end{center}
\end{figure}


\section{Взаимодействие с разработанным ПО}

Взаимодействие с разработанным программным обеспечение осуществляется через графический пользовательский интерфейс. Интерфейс приложения представлен на рисунке \ref{img:gui}.


Графический интерфейс поделен на две части. Первая часть позволяет выбрать и загрузить модель обученную на классификацию и проверить ее работу. Вторая позволяет выбрать и загрузить модель, обученную на детектирование, а также проверить ее работу вместе с моделью классификации.

\section{Тестирование}

\section{Вывод}

В рамках данного раздела были выбраны и реализованы средства распознавания летательных аппаратов. Для этого были выбраны методы аугментации обучающих данных, направленные на увеличение количества и разнообразия данных для обучения модели. 

Были созданы модели для детектирования и классификации летательных аппаратов и реализован алгоритм их обучения. После этого были проведены тесты для оценки качества моделей. Итоговая точность модели классификации на тестовой выборке составила 84 процента. Точность детектирования на тестовой выборке составила 90 процентов.

Кроме того, был создан графический интерфейс, который позволяет пользователям использовать обученную модель для распознавания типов летательных аппаратов. После этого было проведено ручное тестирование обученных моделей, чтобы убедиться в правильности их работы.
\chapter{Исследовательский раздел}

\section{Подбор параметров сети}
Для оценки разработанного метода вводятся следующие метрики качества:
\begin{itemize}
	\item коэффициент попадания -- отношение числа обращений к страницам, которые уже загружены в буфер, к общему числу обращений;
	\item коэффициент совпадения -- отношение количества совпавших с оптимальным алгоритмов кандидатов на замещение с общим числом запросов поиска страниц для вытеснения.
\end{itemize}

Размер скрытых слоев модели подбирался экспериментально.
Графики зависимости коэффициента совпадения в зависимости от эпохи обучения для различных размеров скрытых слоев на обучающей и тестовой выборках представлены на рисунках~\ref{img:test_sizes_train} и~\ref{img:test_sizes_test} соответственно.
\includeimage
{test_sizes_train} % Имя файла без расширения (файл должен быть расположен в директории inc/img/)
{f} % Обтекание (без обтекания)
{H} % Положение рисунка (см. figure из пакета float)
{0.8\textwidth} % Ширина рисунка
{Точность модели для различных размеров скрытых слое на тренировочной выборке} % Подпись рисунка

\includeimage
{test_sizes_test} % Имя файла без расширения (файл должен быть расположен в директории inc/img/)
{f} % Обтекание (без обтекания)
{H} % Положение рисунка (см. figure из пакета float)
{0.8\textwidth} % Ширина рисунка
{Точность модели для различных размеров скрытых слое на тестовой выборке} % Подпись рисунка

Исходя из полученных результатов, настраиваемые параметры модели: $d_z$, $d_b$, $d_h$ и $d_v$ были выбраны равными 448, а $d_f$ -- 32.

\section{Сравнение с аналогами}
Сравнение коэффициентов попадания для разработанного метода и существующих аналогов приведено на рисунке~\ref{img:hits_comp}.

\includeimage
{hits_comp} % Имя файла без расширения (файл должен быть расположен в директории inc/img/)
{f} % Обтекание (без обтекания)
{H} % Положение рисунка (см. figure из пакета float)
{0.8\textwidth} % Ширина рисунка
{Коэффициент попадания в зависимости от числа обращений для различных методов} % Подпись рисунка

Из графиков видно, что коэффициент попадания для разработанного метода в среднем на 0.02 выше чем для алгоритма clock, который в настоящее время используется в PostgreSQL.
Также коэффициент попадания для разработанного метода на 0.08 ниже, чем у оптимального алгоритма.
Таким образом, разработанный метод лучше существующий аналогов, но все еще имеет возможность для улучшения.

\section{Сравнение различных размеров буфера}

Было проведено сравнение точности модели на тренировочной и тестовой выборках при различных размерах буфера: 64, 128 и 256 страниц.
Результаты для тренировочной и тестовой выборках представлена на рисунках~\ref{img:buf_sizes_train_res} и~\ref{img:buf_sizes_test_res} соответственно.

\includeimage
{buf_sizes_train_res} % Имя файла без расширения (файл должен быть расположен в директории inc/img/)
{f} % Обтекание (без обтекания)
{H} % Положение рисунка (см. figure из пакета float)
{0.8\textwidth} % Ширина рисунка
{Коэффициент совпадения в зависимости от эпохи обучения для различных размеров буфера на тренировочной выборке} % Подпись рисунка

\includeimage
{buf_sizes_test_res} % Имя файла без расширения (файл должен быть расположен в директории inc/img/)
{f} % Обтекание (без обтекания)
{H} % Положение рисунка (см. figure из пакета float)
{0.8\textwidth} % Ширина рисунка
{Коэффициент совпадения в зависимости от эпохи обучения для различных размеров буфера на тестовой выборке} % Подпись рисунка

Из полученных графиков видно, что чем больше размер буфера, тем ниже точность модели как на тренировочной, так и на тестовой выборках.
Это связанно с тем, что увеличение размера буфера приводит к увеличению числа обучаемых параметров модели и числа возможных вариантов ответов.
Таким образом, метод может оказаться неэффективным при большом размере буфера.
По умолчанию разделяемых кэш буфер Postgres содержит 128 страниц.
Оценка коэффициентов попадания для разработанного метода и аналогов на таком размере буфера показала, что разработанный метод в среднем на два процента лучше аналогов по этому показателю.

\section{Вывод}
Проведенные исследования позволили определить настраиваемые параметры модели, такие как размеры скрытых слоев $d_z$, $d_b$, $d_h$, $d_v = 448$, $d_f = 32$, обеспечивающие баланс между точностью и вычислительной сложностью. 
Разработанный метод продемонстрировал улучшение коэффициента попадания на 0.02 по сравнению с алгоритмом clock, используемым в PostgreSQL, что подтверждает его практическую эффективность. 
Однако отставание на 0.08 от оптимального алгоритма указывает на потенциал для дальнейшей оптимизации.

Анализ влияния размера буфера показал, что увеличение его объема  приводит к снижению точности модели из-за роста числа обучаемых параметров и вариантов вытеснения.
Результаты, полученные для размера буфера по умолчанию, показали улучшение по сравнению с существующими аналогами по введенным метрикам качества.
Это подтверждает целесообразность внедрения метода в реальные системы с аналогичными настройками.

Таким образом, предложенный метод является перспективным решением для управления замещением страниц в разделяемом кэш буфере PostgreSQL, которое может повысить производительность системы управления базами данных за счет меньшего числа операций, взаимодействующих с диском.
\chapter*{ЗАКЛЮЧЕНИЕ}
\addcontentsline{toc}{chapter}{ЗАКЛЮЧЕНИЕ}

В ходе выполнения выпускной квалификационной работы был спроектирован и разработан метод распознавания летательных аппаратов с аэрофотоснимков с использование нейронных сетей. В ходе выполнения работы были выполнены следующие задачи:
\begin{itemize}
	\item проведен анализ и сравнение существующих методов распознавания летательных аппаратов с аэрофотоснимков;
	\item разработан метод распознавания летательных аппаратов с использованием нейронных сетей;
	\item разработано программное обеспечение, реализующее метод распознавания летательных аппаратов;
	\item исследованы характеристики разработанного метода и влияние на них различных подходов к обучению.
\end{itemize}

Цель работы достигнута.

\makebibliography

\begin{appendices}
	\chapter{Разработанный метод}
	\includelisting
	{read_buffer_ext.c} % Имя файла с расширением (файл должен быть расположен в директории inc/lst/)
	{Модификация ReadBufferExtended} % Подпись листинга
	
	\includelisting
	{read_buffer_ext_con.c} % Имя файла с расширением (файл должен быть расположен в директории inc/lst/)
	{Продолжение листинга А.1} % Подпись листинга
\newpage
	\includelisting
	{page_acc_enc.py} % Имя файла с расширением (файл должен быть расположен в директории inc/lst/)
	{Класс кодировщика запросов обращения к страницам} % Подпись листинга
\newpage
	\includelisting
	{buf_page_enc.py} % Имя файла с расширением (файл должен быть расположен в директории inc/lst/)
	{Класс кодировщика страниц в буфере} % Подпись листинга

	\includelisting
	{page_eviction.py} % Имя файла с расширением (файл должен быть расположен в директории inc/lst/)
	{Модуль выбора страниц для замещения} % Подпись листинга
\newpage
	\includelisting
	{page_eviction_con.py} % Имя файла с расширением (файл должен быть расположен в директории inc/lst/)
	{Продолжение листинга А.5} % Подпись листинга

	\includelisting
	{page_acc_model.py} % Имя файла с расширением (файл должен быть расположен в директории inc/lst/)
	{Модель, реализующая алгоритм замещения страниц} % Подпись листинга

	\includelisting
	{page_acc_model_con.py} % Имя файла с расширением (файл должен быть расположен в директории inc/lst/)
	{Продолжение листинга А.7} % Подпись листинга


	\chapter{Презентация}
	Презентация к выпускной квалификационной работе содержит 20 слайдов.
	
	\newpage
	\section*{Пояснения к слайдам}
	\setcounter{page}{19}
	\textbf{Слайд 8.}
	На вход кодировщика поступают $n$ атрибутов страницы.
	Каждый атрибут может иметь $m_i$ возможных значений, где $i$ -- индекс атрибута.
	Каждый атрибут представляется в виде вектора $a^{(i)}$ размерности $m_i$.
	$W_1^{(i)}$ -- матрица обучаемых весов для скрытого представления i-го атрибута.
	Вектор $z$ -- выходной вектор из сети.
	$W_1$ -- матрица обучаемых весов, при помощи которой получается результирующий вектор из скрытых представлений атрибутов сети.
	$d_f$ и $d_z$ являются настраиваемыми параметрами, которые отвечают за число нейронов, отвечающих за скрытое представление каждого атрибута, и число нейронов на выходном слое соответственно.
	$f$ является конкатенацией векторов скрытых состояний атрибутов страницы, а $b_z$ -- обучаемым вектором.
	
	\textbf{Слайд 9.}
	$[h_{t-1}, z_t]$ -- конкатенация результата работы предыдущего слоя кодировщика истории и скрытого состояния, полученного из кодировщика обращения к странице. $W_f$ и $b_f$ -- матрица и вектор обучаемых весов. $f_t$ -- результат работы фильтра забывания. $i_t$ определяет, какие значения будут сохранены в ячейке. $\hat{C_t}$ -- новые значения кандидатов на попадание в ячейку. $W_i$, $W_C$, $b_i$, $b_c$ -- матрицы и вектора обучаемых весов. $C_t$ -- новое состояние ячейки. $C_{t-1}$ -- состояние ячейки на прошлом шаге. $h_t$ -- результат работы текущего слоя. $C_t$ -- состояние ячейки. $W_o$ и $b_o$ -- матрица и вектор обучаемых весов. Вектора $h_t$ и $C_t$ имеют размерность $d_h$, где $d_h$ -- настраиваемый параметр.
	
	\textbf{Слайд 10.}
	$M$ -- число страниц в буфере. Для каждой j-ой страницы в буфере вычисляется вектор $b_j$ размерности $d_b$, где	$d_b$ является настраиваемым параметром. Для получения скрытого представления каждого атрибута и для вычисления закодированного представления страницы используется одни и те же матрицы весов $W_2^{(i)}$ и $W_2$ для всех страниц в буфере. $f^{(j)}$ -- конкатенация скрытых представлений атрибутов для j-ой страницы в буфере, $b_b$ -- вектор обучаемых весов.
	
	\textbf{Слайд 11.}
	$W_4$ -- матрица обучаемых весов, которая используется для преобразования вектора $h$, полученного из кодировщика истории в вектор контекста размерности $d_v$. $d_v$ является настраиваемым параметром модели.	$W_3$ -- матрица обучаемых весов, которая используется для преобразования закодированного состояния очередной страницы в буфере в вектор размерности $d_v$, который будет использован в функции внимания.
	$v$ -- вектор обучаемых весов, который который используется при вычислении функции внимания. $u_i$ -- результат функции внимания для i-ой страницы в буфере, $p_i$ -- вероятность замещения i-ой страницы.
	
	\textbf{Слайд 14.}
	$E$ -- функция ошибки. $N$ -- число выходов из сети. $t_i$ -- ожидаемое значение на i-ом выходе. $p_i$ -- полученное значение на i-ом выходе.
	
	
\end{appendices}

\end{document}