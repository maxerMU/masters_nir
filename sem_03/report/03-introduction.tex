\chapter*{ВВЕДЕНИЕ}
\addcontentsline{toc}{chapter}{ВВЕДЕНИЕ}

Современные базы данных, такие как PostgreSQL, сталкиваются с постоянно растущими требованиями к производительности и эффективности управления ресурсами.
Одним из ключевых аспектов работы базы данных является управление памятью, включая работу с разделяемым кэш буфером \cite{yuan2022learned}.
Разделяемый кэш буфер используется PostgreSQL для хранения данных, которые часто запрашиваются, что позволяет сократить количество операций чтения с диска и ускорить доступ к данным.

Однако, по мере роста объема данных и количества запросов, разделяемый кэш буфер может оказаться переполненным, что приводит к необходимости вытеснения старых или менее используемых данных. Метод замещения страниц играет важную роль в этом процессе, определяя, какие данные должны быть удалены из кэша, чтобы освободить место для новых.

Целью данной работы является разработка метода замещения страниц в разделяемом кэш буфере PostgreSQL.

Для достижения поставленной цели требуется выполнить следующие задачи:
\begin{itemize}
	\item изложить особенности разрабатываемого метода;
	\item описать основные этапы метода в виде детализированной idef-0 диаграммы;
	\item спроектировать структуру программного обеспечения.
\end{itemize}