\begin{appendices}
	\chapter{Разработанный метод}
	\includelisting
	{read_buffer_ext.c} % Имя файла с расширением (файл должен быть расположен в директории inc/lst/)
	{Модификация ReadBufferExtended} % Подпись листинга
	
	\includelisting
	{read_buffer_ext_con.c} % Имя файла с расширением (файл должен быть расположен в директории inc/lst/)
	{Продолжение листинга А.1} % Подпись листинга
\newpage
	\includelisting
	{page_acc_enc.py} % Имя файла с расширением (файл должен быть расположен в директории inc/lst/)
	{Класс кодировщика запросов обращения к страницам} % Подпись листинга
\newpage
	\includelisting
	{buf_page_enc.py} % Имя файла с расширением (файл должен быть расположен в директории inc/lst/)
	{Класс кодировщика страниц в буфере} % Подпись листинга

	\includelisting
	{page_eviction.py} % Имя файла с расширением (файл должен быть расположен в директории inc/lst/)
	{Модуль выбора страниц для замещения} % Подпись листинга
\newpage
	\includelisting
	{page_eviction_con.py} % Имя файла с расширением (файл должен быть расположен в директории inc/lst/)
	{Продолжение листинга А.5} % Подпись листинга

	\includelisting
	{page_acc_model.py} % Имя файла с расширением (файл должен быть расположен в директории inc/lst/)
	{Модель, реализующая алгоритм замещения страниц} % Подпись листинга

	\includelisting
	{page_acc_model_con.py} % Имя файла с расширением (файл должен быть расположен в директории inc/lst/)
	{Продолжение листинга А.7} % Подпись листинга


	\chapter{Презентация}
	Презентация к выпускной квалификационной работе содержит 20 слайдов.
	
	\newpage
	\section*{Пояснения к слайдам}
	\setcounter{page}{19}
	\textbf{Слайд 8.}
	На вход кодировщика поступают $n$ атрибутов страницы.
	Каждый атрибут может иметь $m_i$ возможных значений, где $i$ -- индекс атрибута.
	Каждый атрибут представляется в виде вектора $a^{(i)}$ размерности $m_i$.
	$W_1^{(i)}$ -- матрица обучаемых весов для скрытого представления i-го атрибута.
	Вектор $z$ -- выходной вектор из сети.
	$W_1$ -- матрица обучаемых весов, при помощи которой получается результирующий вектор из скрытых представлений атрибутов сети.
	$d_f$ и $d_z$ являются настраиваемыми параметрами, которые отвечают за число нейронов, отвечающих за скрытое представление каждого атрибута, и число нейронов на выходном слое соответственно.
	$f$ является конкатенацией векторов скрытых состояний атрибутов страницы, а $b_z$ -- обучаемым вектором.
	
	\textbf{Слайд 9.}
	$[h_{t-1}, z_t]$ -- конкатенация результата работы предыдущего слоя кодировщика истории и скрытого состояния, полученного из кодировщика обращения к странице. $W_f$ и $b_f$ -- матрица и вектор обучаемых весов. $f_t$ -- результат работы фильтра забывания. $i_t$ определяет, какие значения будут сохранены в ячейке. $\hat{C_t}$ -- новые значения кандидатов на попадание в ячейку. $W_i$, $W_C$, $b_i$, $b_c$ -- матрицы и вектора обучаемых весов. $C_t$ -- новое состояние ячейки. $C_{t-1}$ -- состояние ячейки на прошлом шаге. $h_t$ -- результат работы текущего слоя. $C_t$ -- состояние ячейки. $W_o$ и $b_o$ -- матрица и вектор обучаемых весов. Вектора $h_t$ и $C_t$ имеют размерность $d_h$, где $d_h$ -- настраиваемый параметр.
	
	\textbf{Слайд 10.}
	$M$ -- число страниц в буфере. Для каждой j-ой страницы в буфере вычисляется вектор $b_j$ размерности $d_b$, где	$d_b$ является настраиваемым параметром. Для получения скрытого представления каждого атрибута и для вычисления закодированного представления страницы используется одни и те же матрицы весов $W_2^{(i)}$ и $W_2$ для всех страниц в буфере. $f^{(j)}$ -- конкатенация скрытых представлений атрибутов для j-ой страницы в буфере, $b_b$ -- вектор обучаемых весов.
	
	\textbf{Слайд 11.}
	$W_4$ -- матрица обучаемых весов, которая используется для преобразования вектора $h$, полученного из кодировщика истории в вектор контекста размерности $d_v$. $d_v$ является настраиваемым параметром модели.	$W_3$ -- матрица обучаемых весов, которая используется для преобразования закодированного состояния очередной страницы в буфере в вектор размерности $d_v$, который будет использован в функции внимания.
	$v$ -- вектор обучаемых весов, который который используется при вычислении функции внимания. $u_i$ -- результат функции внимания для i-ой страницы в буфере, $p_i$ -- вероятность замещения i-ой страницы.
	
	\textbf{Слайд 14.}
	$E$ -- функция ошибки. $N$ -- число выходов из сети. $t_i$ -- ожидаемое значение на i-ом выходе. $p_i$ -- полученное значение на i-ом выходе.
	
	
\end{appendices}