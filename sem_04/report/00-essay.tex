
%\begin{essay}{}

 
%\end{essay}

\setcounter{page}{3}

\chapter*{РЕФЕРАТ}
Расчетно-пояснительная записка к научно-исследовательской работе содержит  \begin{NoHyper}\pageref{LastPage}\end{NoHyper} страниц, \totfig~иллюстраций, \tottab~таблиц, 6 источников.

Научно-исследовательская работа представляет собой исследование метода замещения страниц в разделяемом кэш буфере PostgreSQL. 
Проведено исследование разработанного метода и выявлена зависимость коэффициентов попадания и совпадения от количества обращений к страницам на тестовой выборке.
Проведено сравнение полученных результатов и значений этих метрик для существующих аналогов.

Разработанный метод замещения страниц может быть использован в СУБД Postgres.
Использование метода позволит повысить коэффициент попадания в разделяемом кэш буфере, что должно привести к уменьшению времени отклика системы.
 
Ключевые слова: страница, замещение, кэш буфер, PostgreSQL.