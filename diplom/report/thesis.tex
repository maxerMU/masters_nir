\documentclass{bmstu}

\bibliography{biblio}

\setlist[itemize]{label={---}}
\setlist[enumerate,2]{label=\arabic*), ref=\arabic*)}

\usepackage{enumitem}

% настройка точек в содержании
\makeatletter
% Настройка плотности точек
\renewcommand{\@dotsep}{1.5} 

% Для глав
\renewcommand*\l@chapter[2]{%
	\@dottedtocline{0}{0em}{1.5em}{\bfseries #1}{#2}}

% Для разделов
\renewcommand*\l@section{\@dottedtocline{1}{1.5em}{2.3em}}
\renewcommand*\l@subsection{\@dottedtocline{2}{3.8em}{3.2em}}
\renewcommand*\l@subsubsection{\@dottedtocline{3}{7.0em}{4.1em}}

\titlespacing*{\section}
{0pt}{5.5ex plus 1ex minus .2ex}{4.3ex plus .2ex}
\titlespacing*{\subsection}
{0pt}{5.5ex plus 1ex minus .2ex}{4.3ex plus .2ex}

\begin{document}


\makethesistitle
{Информатика и системы управления} % Название факультета
{Программное обеспечение ЭВМ и информационные технологии} % Название кафедры
{Метод замещения страниц в разделяемом кэш буфере Postgres с использованием нейронных сетей} % Тема работы
{Мицевич~М.~Д./ИУ7-43М} % Номер группы/ФИО студента (если авторов несколько, их необходимо разделить запятой)
{Тассов~К.~Л.} % ФИО научного руководителя
{} % ФИО консультанта (необязательный аргумент; если консультантов несколько, их необходимо разделить запятой)
{Мальцева~Д.~Ю.} % ФИО нормоконтролера

\setcounter{page}{5}

\chapter*{РЕФЕРАТ}
\addcontentsline{toc}{chapter}{РЕФЕРАТ}

%Расчетно-пояснительная записка к выпускной квалификационной работе содержит  \begin{NoHyper}\pageref{LastPage}\end{NoHyper} страницы, \totfig~иллюстраций, 29 источников, 2 приложения.

%В данной выпускной квалификационной работе представлена разработка метода замещения страниц в СУБД Postgres с использованием нейронных сетей.

%Данная выпускная квалификационная работа посвящена исследованию методов замещения страниц, которые могут быть использованы в разделяемом кэш буфере СУБД Postgres.
%Все методы замещения страниц являются приближением оптимального алгоритма, который предполагает исключать страницу, к которой дольше всего не будет обращений в будущем.
%В данной работе описывается метод замещения страниц, который выбирает страницу для замещения с использованием нейронных сетей.

%В работе рассмотрена задача замещения страниц в СУБД Postgres. 
%Рассмотрены различные методы замещения страниц. 
%Проведен сравнительных анализ различных архитектур нейронных сетей, а также подходов к их обучению. 
%Реализован метод замещения страниц с использование нейронных сетей и проведено исследование точности обученной модели в зависимости от различных значений настраиваемых параметров.
%Проведено сравнение коэффициентов точности и попадания разработанного метода с существующими аналогами.



Расчетно-пояснительная записка к выпускной квалификационной работе «Метод замещения страниц в разделяемом кэш буфере Postgres с использованием нейронных сетей» содержит \begin{NoHyper}\pageref{LastPage}\end{NoHyper}страниц, 4 раздела, \totfig~рисунков, 0 таблиц и список используемых источников из 29 наименований.

Ключевые слова: замещение страниц, рекуррентные сети, Postgres.
	
Объект разработки -- метод замещения страниц в разделяемом кэш буфере Postgres.

Цель работы: разработка метода замещения страниц в разделяемом кэш буфере Postgres с использованием нейронных сетей. 

В первой части работы выполнен анализ существующих методов замещения страниц. 
Изучены принципы работы разделяемого кэш буфера PostgreSQL.
Проведен сравнительный анализ нейронных сетей, подходящих для решения задачи.
Сформулирована цель и формализована постановка задачи в виде IDEF0-диаграммы.

Во второй части разработан метод замещения страниц в разделяемом кэш буфере PostgreSQL.
Описаны основные особенности предлагаемого метода.
Сформулированы ограничения предметной области.
Изложены основные этапы разрабатываемого метода в виде детализированной диаграммы IDEF0 и схем алгоритмов.
Выделены функции и структуры СУБД PostgreSQL, которые используются при работе с кэш буфером.

В третьей части обоснован выбор программных средств реализации метода замещения страниц в разделяемом кэш буфере.
Создана обучающая выборка с помощью логирования обращений к буферу. 
Приведены примеры работы программы. 
Описаны используемые методы тестирования программного обеспечения и приведены его результаты.

В четвертой части проведено исследование разработанного метода и выявлена зависимость коэффициентов попадания и совпадения от количества обращений к страницам на тестовой выборке.
Проведено сравнение полученных результатов и значений этих метрик для существующих аналогов.

Разработанный метод замещения страниц может быть использован в СУБД Postgres. 
Использование метода позволит повысить коэффициент попадания в разделяемом кэш буфере, что должно привести к уменьшению времени отклика системы. 

\maketableofcontents

%\begin{definitions}
	\definition{}{}
\end{definitions}
%\begin{abbreviations}
	\definition{}{}
\end{abbreviations}

\chapter*{ВВЕДЕНИЕ}
\addcontentsline{toc}{chapter}{ВВЕДЕНИЕ}

Современные базы данных, такие как PostgreSQL, сталкиваются с постоянно растущими требованиями к производительности и эффективности управления ресурсами.
Одним из ключевых аспектов работы базы данных является управление памятью, включая работу с разделяемым кэш буфером \cite{yuan2022learned}.
Разделяемый кэш буфер используется PostgreSQL для хранения данных, которые часто запрашиваются, что позволяет сократить количество операций чтения с диска и ускорить доступ к данным.

Однако, по мере роста объема данных и количества запросов, разделяемый кэш буфер может оказаться переполненным, что приводит к необходимости вытеснения старых или менее используемых данных. Метод замещения страниц играет важную роль в этом процессе, определяя, какие данные должны быть удалены из кэша, чтобы освободить место для новых.

Целью данной работы является разработка метода замещения страниц в разделяемом кэш буфере PostgreSQL.

Для достижения поставленной цели требуется выполнить следующие задачи:
\begin{itemize}
	\item изложить особенности разрабатываемого метода;
	\item описать основные этапы метода в виде детализированной idef-0 диаграммы;
	\item спроектировать структуру программного обеспечения.
\end{itemize}
\chapter{Разработка метода}

\section{Разделяемый кэш буфер PostgreSQL}

Кэширование используется в современных компьютерных системах повсеместно: один только процессор имеет три или четыре уровня кэша.
В общем случае кэш нужен для того, чтобы сгладить разницу в производительности двух видов памяти, один из которых в разы быстрее, но меньше по размеру, а другой имеет обратные характеристики размера и времени доступа.
Буферный кэш сохраняет страницы в оперативной памяти, доступ к которой в сотни тысяч раз быстрее, чем к дисковому хранилищу, где содержится вся информация о состоянии базы данных.

В операционной системе также есть дисковый кэш, который решает ту же проблему, поэтому системы управления базами данных обычно стараются избежать двойного кэширования, обращаясь к дискам напрямую, а не через кэш ОС. 
В случае с PostgreSQL это не так: все данные читаются и записываются с помощью обычных файловых операций \cite{shaik2020postgresql}.
При разработке метода важно учитывать, что контроллеры дисковых массивов и даже сами диски также имеют свой собственный кэш.

Буферный кэш является списком буферов, каждый из которых состоит из блока данных и заголовка.
Заголовок содержит:
\begin{itemize}
	\item номер блока страницы;
	\item индикатор того, что страница была изменена, но еще не записана на диск;
	\item число обращений к буферу;
	\item число активных операций или транзакций, которые используют буфер. 
\end{itemize}

При старте все буферы кэша помещаются в список свободных.
Для поиска нужной страницы используется хэш таблица.
В качестве ключа используются номер файла и номер страницы в файле.

При обращении к памяти процесс сначала пытается найти страницу в кэше.
Если она уже загружена, то счетчик обращений в заголовке соответствующего буфера увеличивается на единицу.
До тех пор, пока это счетчик не равен нулю, страница не может быть выгружена из кэша.

Если страница не была найдена в кэше, то она должна быть прочитана с диска в какой-то буфер.
Если список свободных буферов не пуст, то будет взят первый буфер из него, иначе требуется выбрать буфер, который будет вытеснен из кэша.

В PostgreSQL для выбора кандидата на замещение анализируются два счетчика -- число обращений и количество использований.
Алгоритм часы поочередно проходит по всем буферам в кэше и, если оба счетчика равны нулю, то текущий буфер будет замещен, иначе оба счетчика уменьшаются на единицу.
Для избежания большого числа проходов по всем буферам в поисках кандидата на замещение по умолчанию счетчики не могут быть больше пяти.

Когда кандидат на замещение найден, счетчик использований ассоциированного с ним буфера устанавливается в 1, чтобы другие процессы не могли его использовать.
Если буфер содержит не записанную диск информацию, то запускается фоновый процесс переписывания страницы на диск.
После этого новая страница загружается в буфер и для нее выставляется счетчик обращений в единицу.

\section{Постановка задачи}
Пусть, размер разделяемого кэш буфера в СУБД равен $B$.
Тогда в каждый момент времени $t$, когда возникает новое обращение к странице в памяти, СУБД пытается найти нужный буфер в кэше.
Если он отсутствует и кэш заполнен, то при помощи алгоритма замещения должна быть выбрана страница, которая будет вытеснена из буфера.
Результатом работы метода является $a^t \in \{0, 1, ..., B-1\}$ -- индекс страницы, которая будет вытеснена.

Для оценки разработанного алгоритма будут использованы следующие метрики качества:
\begin{itemize}
	\item коэффициента попадания -- отношение числа обращений к страницам, которые уже загружены в буфере, к общему числу обращений;
	\item коэффициент совпадения -- показывает отношение количества совпавших с оптимальным алгоритмов кандидатов на замещение с общим числом запросов поиска буфера для вытеснения.
\end{itemize}

\section{Описание модели}
Разрабатываемая модель состоит из четырех компонентов \cite{deepBM}:
\begin{enumerate}
	\item Кодировщик запроса доступа к странице -- полносвязная нейронная сеть, которая на вход получает атрибуты запрашиваемой страницы, а на выходе выдает вектор $e_t$ размерности $d_e$.
	\item Кодировщик страниц в буфере -- полносвязная нейронная сеть, на вход которой поступают атрибуты страниц в буфере и результатом которой является список векторов для каждой страницы в буфере. Размерность каждого вектора равна $d_b$.
	\item Кодировщик истории обращений к страницам;
	\item Модуль выбора страницы для замещения. 
\end{enumerate}

\textbf{Кодировщик запроса доступа к странице.}
При каждом запросе страницы из запроса извлекаются следующие параметры:
\begin{itemize}
	\item $rel\_id$ -- идентификатор отношения;
	\item $is\_local\_temp$ -- флаг, который отвечает показывает является отношение временным в текущей сессии;
	\item $fork\_num$ -- тип физического хранилища;
	\item $blk\_num$ -- номер блока в файле;
	\item $mode$ -- метод чтения страницы;
	\item $rel\_am$ -- тип индекса;
	\item $rel\_file\_node$ -- идентификатор физического хранилища;
	\item $has\_index$ -- флаг наличия индекса;
	\item $rel\_persistence$ -- тип хранения объекта (постоянный, временный, объект, который не ведет журнал транзакций);
	\item $rel\_kind$ -- тип отношения: обычная таблица, индекс, последовательность, таблица с специальной техникой хранения больших объектов, отображение, сложный тип, внешняя таблица, разделенная таблица, разделенный индекс;
	\item $rel\_natts$ -- список пользовательских атрибутов;
	\item $relfrozenxid$ -- идентификатор транзакции, который указывает на момент, когда все старые версии строк в данной таблице были заморожены;
	\item $relminmxid$ -- минимальный идентификатор многоверсионной транзакции для данной таблицы.
\end{itemize}

Помимо описанных выше атрибутов к входным параметрам модели добавляется индекс страницы в буфере, если она там есть, или число равное размеру буфера, как признак отсутствия страницы в нем.

Для представления категориальных данных применяется техника однозначного кодирования \cite{onecode}, при которой каждая категория представляет бинарным вектором, размер которого совпадает с числом категорий.
Такой подход позволяет представить категории в виде дискретных типов без введения отношения порядка над ними.

Входные данные поступают на вход полносвязной нейронной сети с одним скрытым слоем \cite{митина2021перцептрон}.
Результатом работы сети является вектор признаков размерностью $d_e$.

\textbf{Кодировщик страниц в буфере.}
Кодировщик страниц в буфере работает аналогично кодировщику запроса доступа.
Единственным отличием является другой набор обучаемых весов и входных параметров.
На вход нейронной сети помимо атрибутов каждой страницы также поступает закодированное представление индекса в буфере.
Для кодирования индекса сначала применяется техника однозначного кодирования, а затем преобразование его в вектор размерности $d_f$ путем умножения на обучающиеся матрицы смежности.

\textbf{Кодировщик истории обращения к страницам.}
Для запоминания истории запросов к буферу использована рекуррентная нейронная сеть LSTM \cite{yu2019review}.
На вход сети поступает вектор из кодировщика запросов доступа страницы.
Результатом работы сети является вектор $h^t$ размера $d_h$, который описывает историю обращений.

Сеть LSTM состоит из четырех компонентов.
Ключевой из них -- состояние ячейки, которая переходит между повторяющимися модулями сети, подвергаясь преобразованиям.
Три других компонента отвечают за забывание прошлого состояния ячейки, обновление состояния на основе входных данных и выхода из прошлого модуля, а также за получение выходного значения из текущего блока.

Первый компонент нужен для определения того, какую часть информации можно выбросить из состояния ячейки.
На вход к нему поступают входные данные в текущий блок и выходной вектор из прошлого модуля.
На выходе при помощи сигмоидного фильтра для каждого значения в состоянии ячейки вычисляется число от 0 до 1, после чего происходит поточечное умножение чисел в ячейке на полученных из фильтра забывания значения.

Задача следующего компонента -- определить какая новая информация будет сохранена в ячейке.
Для этого сначала при помощи сигмоидного входного фильтра определяются значения, которые будут сохранены в ячейке, а затем с использованием слоя гиперболического тангенса вычисляются новые значения кандидатов на попадание в ячейку.
После этих операций выполняется поточечное суммирование элементов в ячейке с полученными кандидатами.

Задача последнего компонента -- определить, какая информация будет на выходе из текущего модуля.
Для этого используется поточечное умножение текущего состояния ячейки, пропущенного через фильтр гиперболического тангенса, и входных данных, объединенных с выходом из прошлого модуля и прошедших через сигмоидный фильтр.

\textbf{Модуль выбора страницы для замещения.}
Для принятия решения о том, какая страница будет вытеснена из буфера используется еще одна полносвязная нейронная сеть.
На вход сети поступают выходной вектор из кодировщика истории обращения к страницам и список выходных векторов из кодировщика страниц в буфере.
Результатом работы сети является вектором размером, совпадающим с количеством элементов в кэше, где каждый элемент показывает вероятность того, что та или иная страница должна быть замещена.
Для замещения будет выбрана страница с наибольшей вероятностью.

\section{Получение обучающей выборки}
Для создания обучающей выборки необходимо получить последовательность обращений к страницам в буфере.
Имея такую последовательность, каждый раз, когда требуется исключить страницу из буфера, можно выбрать кандидата при помощи оптимального алгоритма.

Такая последовательность была получена путем модификации исходного кода PostgreSQL и добавления логгирования в функцию ReadBufferExtended, которая вызывается каждый раз, когда необходимо прочитать страницу из буфера.

Для получения информации о свойствах запрашиваемого отношения используется поле rd\_rel структуры Relation, которая передается в функцию в качестве входного аргумента.
Тип физического хранилища, номер блока в файле и метод чтения страницы также являются входными аргументами функции.
Для проверки того, является ли отношение временным, анализируется поле backend у структуры RelFileLocatorBackend.
Для проверки того, загружена ли страница в буфер, анализируется возвращаемое значение из текущей реализации функции замещения страниц.

\section{Детализированная IDEF-0}

Детализированная IDEF-0 диаграмма представлена на рисунке \ref{img:idef0}.

\includeimage
{idef0} % Имя файла без расширения (файл должен быть расположен в директории inc/img/)
{f} % Обтекание (без обтекания)
{H} % Положение рисунка (см. figure из пакета float)
{\textwidth} % Ширина рисунка
{Детализированная IDEF-0 диаграмма} % Подпись рисунка

Запрос обращения к странице проходит через кодировщик запросов и попадает в блок, который отвечает за историю обращений.
Текущая история обращений к страницам и закодированные страницы, которые в данных момент находятся в буфере, попадают на вход блока, который отвечает за выбор страницы для замещений.

\section{Структура ПО}

Структура ПО приведена на рисунке \ref{img:structure}.
Каждый блок, представленный в детализированной idef0 диаграмме, является отдельным модулем в структуре программного обеспечения.
Также отдельный модуль отвечает за получения обучающей выборки при моделировании типовой нагрузки на СУБД.

\includeimage
{structure} % Имя файла без расширения (файл должен быть расположен в директории inc/img/)
{f} % Обтекание (без обтекания)
{H} % Положение рисунка (см. figure из пакета float)
{\textwidth} % Ширина рисунка
{Структура ПО} % Подпись рисунка
\chapter{Конструкторский раздел}

\section{Входные данные}
На вход методу подается атрибуты страницы, к которой происходит обращение, и атрибуты всех страниц, которые уже находятся в буфере.

Для сохранения истории обращений к кэш буферу в функцию ReadBufferExtended был добавлен вызов функции печати в лог файл атрибутов страницы, к которой идет обращение.
Эта функция вызывается, каждый раз, когда необходимо прочитать страницу из буфера.

Для каждой страницы извлекается следующий набор атрибутов:
\begin{itemize}
	\item идентификатор отношения;
	\item номер страницы в файле;
	\item наличие индекса;
	\item позиция в буфере.
\end{itemize}

Если в момент обращения к странице она не находится в буфере, то позиция задается размером буфера.

\section{Проектирование метода}
Детализированная IDEF0 диаграмма метода замещения страниц уровня A0 приведена на рисунке~\ref{img:idef0A1}.
\includeimage
{idef0A1} % Имя файла без расширения (файл должен быть расположен в директории inc/img/)
{f} % Обтекание (без обтекания)
{H} % Положение рисунка (см. figure из пакета float)
{\textwidth} % Ширина рисунка
{Детализированная IDEF0 диаграмма} % Подпись рисунка

\textbf{Кодировщик запроса обращения к странице} отвечает за скрытое представление атрибутов страницы, к которой происходит очередное обращение.
Схема кодировщика запроса обращения к странице изображена на рисунке~\ref{img:page_acc}.
\includeimage
{page_acc} % Имя файла без расширения (файл должен быть расположен в директории inc/img/)
{f} % Обтекание (без обтекания)
{H} % Положение рисунка (см. figure из пакета float)
{0.6\textwidth} % Ширина рисунка
{Схема кодировщика запроса обращения к странице} % Подпись рисунка

На вход кодировщика поступают $n$ атрибутов страницы.
Каждый атрибут может иметь $m_i$ возможных значений, где $i$ -- индекс атрибута.
Каждый атрибут представляется в виде вектора $a^{(i)}$ размерности $m_i$.
Для категориальных данных используется техника однозначного кодирования, а для числовых -- применяется хэш функция и к полученному результату применяется техника однозначного кодирования.
$W_1^{(i)}$ -- матрица обучаемых весов для скрытого представления i-го атрибута.
Вектор $z$ -- выходной вектор из сети.
$W_1$ -- матрица обучаемых весов, при помощи который получается результирующий вектор из скрытых представлений атрибутов сети.
В качестве функции активации на последнем слое используется функция Relu.
$d_f$ и $d_z$ являются настраиваемыми параметрами, которые отвечают за число нейронов, отвечающий за скрытое представление каждого атрибута, и число нейронов на выходном слое соответственно.

Работу сети можно описать с помощью выражений
(\ref{formula:page_enc_1}) -~(\ref{formula:page_enc_3}):
\begin{equation}\label{formula:page_enc_1}
	f^{(i)} = a^{(i)}W_1^{(i)} i \in \{1;n\},
\end{equation}

\begin{equation}\label{formula:page_enc_2}
	f = [f^{(1)}, f^{(2)}, ..., f^{(n)}],
\end{equation}

\begin{equation}\label{formula:page_enc_3}
	z = ReLU(W_1f^T + l_1),
\end{equation}
где $f$ является конкатенацией векторов скрытых состояний атрибутов страницы, а $l_1$ -- обучаемым вектором.

\textbf{Кодировщик страниц в буфере} нужен для скрытого представления каждой страницы в буфере.
Схема кодировщика представлена на рисунке~\ref{img:buf_page_enc}
\includeimage
{buf_page_enc} % Имя файла без расширения (файл должен быть расположен в директории inc/img/)
{f} % Обтекание (без обтекания)
{H} % Положение рисунка (см. figure из пакета float)
{0.8\textwidth} % Ширина рисунка
{Схема кодировщика страниц в буфере} % Подпись рисунка

На вход кодировщика поступают $M$ страниц из буфера.
Каждая страница представляется в виде $n$ атрибутов.
Процесс обработки атрибутов для каждой страницы такой же, как и в кодировщике запроса обращения к странице.
Для каждой i-ой страницы в буфере вычисляется вектор $b_i$.
$d_b$ является настраиваемым параметром, который отвечает за размерность векторов $b_i$.

Для получения скрытого представления каждого атрибута и для вычисления закодированного представления страницы используется одни и те же матрицы весов $W_2^{(i)}$ и $W_2$ для всех страниц в буфере.
За счет этого матрицы весов не привязаны к конкретной позиции страницы в буфере и истории страниц на этой позиции.
При обратном распространении ошибки влияние веса из матрицы $W_2$ будет учитываться для всех векторов $b_i$.

Обозначим результат работы сумматора нейрона на выходном слое как $s_{ij}$.
Индексация в матрице $s$ совпадает с матрицей $b$.
Тогда для вычисления ошибки по весу $w_{ij}$ из матрицы $W_2$ на ребре, которое соединяет j-ый нейрон из второго слоя и i-ый нейрон из выходного слоя, используется выражение~(\ref{formula:buf_page_enc_err}):
\begin{equation}\label{formula:buf_page_enc_err}
	\frac{\delta E}{\delta w_{ij}} = \sum\limits_{k=1}^{M}\frac{\delta E}{\delta b_{ki}}\frac{\delta b_{ki}}{\delta s_{ki}} \frac{\delta s_{ki}}{\delta w_ij},
\end{equation}
где $E$ -- функция ошибки, $\frac{\delta E}{\delta b_{ki}}$ -- ошибка полученная со следующего слоя.

Функционирование кодировщика определяется выражениями, приведенными ниже
~{(\ref{formula:buf_page_enc_1}) - (\ref{formula:buf_page_enc_3})}:
\begin{equation}\label{formula:buf_page_enc_1}
	f^{(j,i)} = a^{(j,i)}W_2^{(i)}, j \in \{1;M\}, i \in \{1;n\},
\end{equation}

\begin{equation}\label{formula:buf_page_enc_2}
	f^{(j)} = [f^{(j,1)}, f^{(j,2)}, ..., f^{(j,n)}],
\end{equation}

\begin{equation}\label{formula:buf_page_enc_3}
	b_j = ReLU(W_2f^{(j)T} + b_b),
\end{equation}
где $f^{(j)}$ -- конкатенация скрытых представлений атрибутов для j-ой страницы в буфере, $b_j$ -- скрытое представление этой страницы, $l_2$ -- вектор обучаемых весов.

\textbf{Кодировщик истории обращений.} Для обновления истории обращений используется сеть LSTM.
На вход сети поступают результат работы кодировщика обращения к странице, предыдущий результат кодировщика истории обращений и предыдущее состояние ячейки.

Функционирование кодировщика определяется выражениями приведенными ниже
(\ref{formula:lstm_enc_1}) - (\ref{formula:lstm_enc_6}):
\begin{equation}\label{formula:lstm_enc_1}
	f_t = \sigma(W_f[h_{t-1}, z_t] + b_f),
\end{equation}

\begin{equation}\label{formula:lstm_enc_2}
	i_t = \sigma(W_i[h_{t-1}, z_t] + b_i),
\end{equation}

\begin{equation}\label{formula:lstm_enc_3}
	\hat{C_t} = \tanh(W_C[h_{t-1}, z_t] + b_C),
\end{equation}

\begin{equation}\label{formula:lstm_enc_4}
	C_t = f_t \cdot C_{t-1} + i_t \cdot \hat{C_t},
\end{equation}

\begin{equation}\label{formula:lstm_enc_5}
	o_t = \sigma(W_o[h_{t-1}, z_t] + b_o),
\end{equation}

\begin{equation}\label{formula:lstm_enc_6}
	h_t = o_t \cdot \tanh(C_t),
\end{equation}
где:
\begin{itemize}
	\item $[h_{t-1}, z_t]$ -- конкатенация результата работы предыдущего слоя кодировщика истории и скрытого состояния, полученного из кодировщика обращения к странице;
	\item $W_f$ и $b_f$ -- матрица и вектор обучаемых весов;
	\item $f_t$ -- результат работы фильтра забывания;
	\item $i_t$ определяет, какие значения будут сохранены в ячейке;
	\item $\hat{C_t}$ -- новые значения кандидатов на попадание в ячейку;
	\item $W_i$, $W_C$, $b_i$, $b_c$ -- матрицы и вектора обучаемых весов;
	\item $C_t$ -- новое состояние ячейки;
	\item $C_{t-1}$ -- состояние ячейки на прошлом шаге;
	\item $h_t$ -- результат работы текущего слоя;
	\item $C_t$ -- состояние ячейки;
	\item $W_o$ и $b_o$ -- матрица и вектор обучаемых весов.
\end{itemize}
Вектора $h_t$ и $C_t$ имеют размерность $d_h$, где $d_h$ -- настраиваемый параметр.

\textbf{Модуль выбора страниц для замещения.}
На вход модуля поступают результаты работы кодировщика страниц в буфере и кодировщика истории обращений.
Для выбора страницы, которая будет удалена из буфера используется указательная нейронная сеть с механизмом внимания~\cite{vinyals2015pointer}.

Нейронные сети с механизмом внимания -- это архитектуры, которые позволяют моделям динамически фокусироваться на наиболее релевантных частях входных данных при обработке информации.
Этот подход нашел применения в областях обработки естественного языка, компьютерного зрения и других задач, где важно учитывать контекст и зависимости между элементами последовательности.
В модуле выбора страниц для замещения контекстом является результат работы кодировщика истории, а элементами последовательности -- результаты работы кодировщика страниц в буфере.

Указательные сети -- архитектура сетей с механизмом внимания, предназначенная для решения задач, где выходные элементы представляют собой позиции в входной последовательности.
В указательных сетях механизм внимания используется как указатель на один из элементов входной последовательности, а не для создания контекстного вектора, как в классических моделях с механизмом внимания.

Схема модуля выбора страниц для замещения представлена на рисунке~\ref{img:decision_maker}
\includeimage
{decision_maker} % Имя файла без расширения (файл должен быть расположен в директории inc/img/)
{f} % Обтекание (без обтекания)
{H} % Положение рисунка (см. figure из пакета float)
{0.8\textwidth} % Ширина рисунка
{Схема модуля выбора страниц для замещения} % Подпись рисунка

$W_4$ -- матрица обучаемых весов, которая используется для преобразования вектора $h$, полученного из кодировщика истории в вектор контекста размерности $d_v$. $d_v$ является настраиваемым параметром модели.

$W_3$ -- матрица обучаемых весов, которая используется для преобразования закодированного состояния очередной страницы в буфере в вектор размерности $d_v$, который будет использован в функции внимания.

$v$ -- вектор обучаемых весов, который который используется при вычислении функции внимания.

Функционирование модуля выбора страниц для замещения определяется выражениями~(\ref{formula:evict_1}) -~(\ref{formula:evict_3}):
\begin{equation}\label{formula:evict_1}
	u_i = v * \tanh(W_3 b_i^T + W_4 h^T), i \in \{1; M\},
\end{equation}

\begin{equation}\label{formula:evict_2}
	p_i = SoftMax(u_i),
\end{equation}

\begin{equation}\label{formula:evict_3}
	r = \arg\max_{i} p_i,
\end{equation}
где $M$ -- число страниц в буфере, $u_i$ -- результат функции внимания для i-ой страницы в буфере, $\arg\max_{i} p_i$ -- функция, которая возвращает индекс максимального элемента в последовательности, $r$ -- результат работы спроектированного метода замещения страниц.

Схемы алгоритмов обучения нейронных сетей и прохождения одной эпохи приведены на рисунках~\ref{img:training} и~\ref{img:training_epoch} соответственно.
\includeimage
{training} % Имя файла без расширения (файл должен быть расположен в директории inc/img/)
{f} % Обтекание (без обтекания)
{H} % Положение рисунка (см. figure из пакета float)
{0.5\textwidth} % Ширина рисунка
{Схема алгоритма обучения нейронной сети} % Подпись рисунка

\includeimage
{training_epoch} % Имя файла без расширения (файл должен быть расположен в директории inc/img/)
{f} % Обтекание (без обтекания)
{H} % Положение рисунка (см. figure из пакета float)
{0.6\textwidth} % Ширина рисунка
{Схема алгоритма прохождения одной эпохи} % Подпись рисунка

\section{Структура программного обеспечения}

Программное обеспечение состоит из пяти модулей:
\begin{itemize}
	\item модуль получения обучающей выборки;
	\item кодировщик запросов обращения к страницам;
	\item кодировщик истории запросов обращения к страницам;
	\item кодировщик страниц в буфере;
	\item модуль выбора страниц для замещения.
\end{itemize}

Структурная схема взаимодействия модулей разрабатываемого программного обеспечения представлена на рисунке~\ref{img:modules}.

\includeimage
{modules} % Имя файла без расширения (файл должен быть расположен в директории inc/img/)
{f} % Обтекание (без обтекания)
{H} % Положение рисунка (см. figure из пакета float)
{0.8\textwidth} % Ширина рисунка
{Структура программного обеспечения} % Подпись рисунка

Модуль получения обучающей выборки нужен для обработки лог файла и создания обучающей и тестовой выборок.
Остальные модули предназначены для обучения и взаимодействия с уже обученными моделями.

\section{Набор обучающих данных}
Для имитации нагрузки на СУБД и получения истории обращений к страницам был использован тестовый сценарий TPC-C~\cite{leutenegger1993modeling}.

TPC-C -- это стандартный тест для оценки производительности систем, обрабатывающих транзакции в режиме реального времени. 
Он имитирует работу оптового поставщика с распределённой сетью складов и предназначен для тестирования СУБД на реалистичных сценариях высокой нагрузки.

Для обработки нагрузки создается база данных, состоящая из следующих таблиц:
\begin{itemize}
	\item Warehouse (склады);
	\item District (регионы складов);
	\item Customer (клиенты);
	\item Order (заказы);
	\item Order-Line (позиции заказов);
	\item Item (товары);
	\item Stock (запасы на складах);
	\item History (история платежей).
\end{itemize}

TPC-C включает 5 типов транзакций, имитирующих реальные бизнес-операции:
\begin{enumerate}
	\item NewOrder (45\%): создание нового заказа (вставка данных в таблицы Order, Order-Line, обновление Stock).
	\item Payment (43\%): Обработка платежа (обновление Customer, Warehouse, District, вставка в History).
	\item Delivery (4\%): Доставка заказа (удаление из Order, обновление Customer).
	\item OrderStatus (4\%): Проверка статуса заказа (выборка из Order, Order-Line, Customer).
	\item StockLevel (4\%): Проверка уровня запасов (выборка из Stock).
\end{enumerate}

С помощью тестовой нагрузки было получено 6 554 959 обращений к страницам.
Выборка была поделена на тренировочную и тестовую в отношении семь к трем.

Корреляция между атрибутами страниц, посчитанная на всей выборке, представлена на рисунке~\ref{img:correlation}.

\includeimage
{correlation} % Имя файла без расширения (файл должен быть расположен в директории inc/img/)
{f} % Обтекание (без обтекания)
{H} % Положение рисунка (см. figure из пакета float)
{0.8\textwidth} % Ширина рисунка
{Корреляция между атрибутами страниц} % Подпись рисунка

Корреляция вычислялась по формуле Пирсона~(\ref{formula:pirson}):
\begin{equation}\label{formula:pirson}
	c = \frac{\sum\limits_{i=1}^{n} (x_i - \overline{x}) (y_i - \overline{y})}{\sqrt{\sum\limits_{i=1}^{n}(x_i - \overline{x})^2} \sqrt{\sum\limits_{i=1}^{n}(y_i - \overline{y})^2}},
\end{equation}
где $n$ -- число элементов в последовательности, $x_i$ и $y_i$ -- элементы последовательностей, между которыми считается корреляция, $\overline{x}$ и $\overline{y}$ -- средние значения элементов последовательностей.

Количество замещений страницы оптимальным алгоритмом в зависимости от индекса страницы в буфере для тренировочной и тестовой выборок приведено на рисунках~\ref{img:evict_freq_train} и~\ref{img:evict_freq_test} соответственно.

\includeimage
{evict_freq_train} % Имя файла без расширения (файл должен быть расположен в директории inc/img/)
{f} % Обтекание (без обтекания)
{H} % Положение рисунка (см. figure из пакета float)
{0.8\textwidth} % Ширина рисунка
{Количество замещений страницы по индексу на тренировочной выборке} % Подпись рисунка

\includeimage
{evict_freq_test} % Имя файла без расширения (файл должен быть расположен в директории inc/img/)
{f} % Обтекание (без обтекания)
{H} % Положение рисунка (см. figure из пакета float)
{0.8\textwidth} % Ширина рисунка
{Количество замещений страницы по индексу на тестовой выборке} % Подпись рисунка


\section{Вывод}

В данном разделе были определены ограничения, которые накладываются на входные данные, и требования, которые предъявляются к разрабатываемому программному обеспечению.

Была детализирована IDEF-0 диаграмма уровня А0, описанная в разделе формализованной постановки задачи, а также было проведено разбиение программного обеспечения на модули.
Задача выбора страниц для замещения из буфера была разбита на четыре подзадачи: кодирование запроса обращения к странице, обновление истории обращений, кодирование страниц в буфере, выбор страниц для замещения на основе контекста.

Для решения каждой подзадачи была спроектирована архитектура нейронной сети.
Для обеспечения временных зависимостей при обновлении истории была выбрана сеть LSTM.
Для выбора страницы для замещения была спроектирована указательная сеть с механизмом внимания, которая выбирает одну из страниц в буфере для замещения.
Была определена схема алгоритма обучения составленных нейронных сетей.

Для генерации реалистичной нагрузки использован тестовый сценарий TPC-C, что позволило получить шесть с половинной миллионов обращений к страницам.
Распределение замещений страниц на тренировочной и тестовой выборках подтвердило сбалансированность данных.
\chapter{Технологический раздел}

\section{Средства реализации программного обеспечения}

В качестве языка программирования был выбран Python \cite{Python}. Данный выбор обусловлен тем, что Python имеет множество библиотек, таких как TensorFlow, Keras, PyTorch, которые предоставляют множество инструментов для создания и обучения нейронных сетей.

В качестве библиотеки для создания нейронной сети была выбрана библиотека PyTorch \cite{PyTorch} версии 2.0.0, так как она имеет следующие возможности и инструменты:
\begin{itemize}
	\item динамический граф вычислений, использование которого облегчает отладку моделей;
	\item возможность переноса вычислений на GPU;
	\item набор инструментов для создания различных слоев, из которых складывается архитектура нейронной сети;
	\item API на языке C++, что позволяет обучить модель с использованием интерпретируемого языка Python, а использовать ее на компилируемом языке C++.
\end{itemize}

Для работы с большими данными была выбрана библиотека numpy версии 1.21.0, так как она использует оптимизированный код на C, что позволяет выполнять вычисления быстрее, чем с использованием чистого Python, а также потому что классы этой библиотеки интегрируются с библиотекой PyTorch, которая используется для создания нейронной сети.

Для анализа входных данных использовались библиотеки matplotlib, pandas и seaborn.
Pandas поддерживает чтение данных из различных форматов данных в структуру DataFrame -- таблицу с индексами и метками, а также совместимость с другими библиотеками, такими как numpy, для передачи данных.
Matplotlib имеет возможность создания различных видов графиков: линейные, столбчатые, гистограммы, а также имеет интеграцию с Jupyter Notebook для интерактивной визуализации.
Seaborn позволяет строить карту корреляций для элементов Dataframe.

Для создания графического интерфейса был использован фреймворк PyQT \cite{qt}, так как он является кроссплатформенным, имеет собственную библиотеку стандартных виджетов, имеет документацию по всем структурам, а также поддерживает последние стандарты языка Python и имеет дополнительные утилиты такие как QtDesigner, которые упрощают создание графических интерфейсов.

\section{Разработка программного комплекса}

Для создания обучающей выборки необходимо провести следующую временную модификацию в СУБД PostgreSQL: нужно найти функцию, которая вызывается каждый раз при обращении к страницам в буфере и добавить в эту функцию вызов функции для записи информации о странице в специальный файл.

Для управления разделяемым кэш буфером используются следующие функции:
\begin{enumerate}
	\item ReadBuffer -- захватывает буфер, увеличивает его pin count, и загружает страницу в буфер, если её там нет.
	\item ReleaseBuffer -- уменьшает pin count буфера, освобождая его для возможного повторного использования.
	\item LockBuffer, LockBufferForCleanUp, ConditionalLockBufferForCleanUp -- управление блокировками.
	\item BgBufferSync -- фоновая запись измененных буферов на диск.
	\item CheckPoint -- управление контрольными точками.
	\item MarkBufferDirty -- помечает буфер, как требующий записи на диск.
	\item FlushBuffer -- сбрасывает содержимое буфера на диск.
	\item BufferAlloc -- выделяет буфер для новой страницы.
\end{enumerate}

Функция ReadBuffer принимает указатель на структуру отношения, для которого читается блок данных, и номер блока для чтения.
Внутри функции ReadBuffer происходит вызов функции ReadBufferExtended, которая дополнительно принимает имя физического хранилища, способ чтения и стратегию управления буфером.
Вся логика обращений и изменений разделяемого кэш буфера написана внутри функции ReadBufferExtended.
Эта функция вызывается с различными аргументами еще из 66 мест, поэтому именно в нее надо добавлять запись в лог файл с информацией об обращении к странице.
Модифицированный код функции ReadBufferExtended приведен в листинге \ref{lst:read_buffer_ext.c} (приложение А).

Для реализации кодировщика запросов обращения к странице был разработан класс PageAccEncoder.
Класс является наследником класса Module из библиотеки PyTorch:
\begin{itemize}
	\item слои эмбеддиногов, которые преобразуют атрибуты страницы в некоторые скрытые вектора;
	\item полносвязная сеть, которая объединяет результаты эмбедингов для каждого атрибута страницы и создает скрытое представление страницы.
\end{itemize}

Для целочисленных атрибутов применяется модель HashEmbedding, которая сначала вычисляет хэш функцию, а затем к полученному значению применяет слой эмбеддинга.
Для категориальных атрибутов он применяется сразу.

Полносвязный слой является последовательным применением Linear блока и функции активации ReLU.
Реализация класса PageAccEncoder представлена в листинге \ref{lst:page_acc_enc.py} (приложение А).

Для реализации кодировщика страниц в буфере был разработан класс PageBufferEncoder.
Этот класс является наследником PageAccEncoder, так как с каждой страницей в буфере требуется выполнить те же действия, что и с страницей, к которой идет новое обращение.
Для параллелизации и ускорения процессов взаимодействия с моделью перед применением слое модели для всех атрибутов страниц в буфере вызывается функция torch.stack, которая нужная для конкатенации тензоров вдоль новой оси.
Реализация класса PageBufferEncoder представлена в листинге \ref{lst:buf_page_enc.py}.

В качестве кодировщика истории обращений используется библиотечная модель LSTM.

Для реализации модуля выбора страниц для замещения был разработан класс PageEviction, который наследуется от класса Module из библиотеки PyTorch.
Модель состоит из трех линейных слоев:
\begin{itemize}
	\item attention\_page -- преобразует скрытое представление страницы в буфере в пространство внимания;
	\item attention\_context -- преобразует результат работы кодировщика истории в это же пространство;
	\item attention\_v -- вычисляет итоговые оценки внимания на основе объединенных признаков из пространства внимания.
\end{itemize}
Реализация класса PageEviction представлена в листинге \ref{lst:page_eviction.py}.

Класс PageAccModel объединяет в себе все 4 описанных выше модуля.
Процесс обработки входных данных и выбора страницы для замещения состоит из четырех шагов:
\begin{itemize}
	\item кодирование запроса обращения к странице -- используется модель PageAccEncoder;
	\item обновление истории обращений -- используется модель LSTM;
	\item кодирование страниц в буфере -- используется модель PageBufferEncoder;
	\item выбор страницы для замещения -- используется модуль PageEviction.
\end{itemize}
Реализация класса PageAccModel представлена в листинге \ref{lst:page_acc_model.py}.

\section{Обучение и тестирование модели}

Обучение модели проводилось на машине с процессором Intel Core i9-10900, 64 гигабайтами оперативной памяти и графической картой NVIDIA GeForce RTX 3080 с 16 гигабайтами памяти типа GDDR6.

В качестве оптимизатора функции потерь был выбран Adam, так как он автоматически адаптирует скорость обучения для каждого параметра в зависимости от его градиента, что позволяет более эффективно использовать скорость обучения и ускоряет сходимость.

Обучение модели проводилось на протяжении 100 эпох.
После прохождения каждой эпохи веса модели сохранялись в файл и вычислялась точность модели на тестовой выборке.
Была выбрана модель с наивысшей точностью на тестовой выборке.

Графики зависимостей точности модели на тестовой и обучающей выборках от номера эпохи обучения приведены на рисунке \ref{img:training_512}.

\includeimage
{training_512} % Имя файла без расширения (файл должен быть расположен в директории inc/img/)
{f} % Обтекание (без обтекания)
{H} % Положение рисунка (см. figure из пакета float)
{0.8\textwidth} % Ширина рисунка
{Точность при обучении модели на тренировочной и тестовой выборках} % Подпись рисунка

Наивысшая точность модели была получена на 97 эпохе -- 44.2 процента.
Точность на обучающей выборке составила 63 процента.

\section{Взаимодействие с разработанным ПО}

Взаимодействие с разработанным программным обеспечение осуществляется через графический пользовательский интерфейс. Интерфейс приложения представлен на рисунке \ref{img:gui}.

\includeimage
{gui} % Имя файла без расширения (файл должен быть расположен в директории inc/img/)
{f} % Обтекание (без обтекания)
{H} % Положение рисунка (см. figure из пакета float)
{0.8\textwidth} % Ширина рисунка
{Интерфейс разработанного приложения} % Подпись рисунка


Графический интерфейс поделен на две части.
Первая часть позволяет выбрать и загрузить выборку для тестирования, обученную модель и оптимальные результаты, а также выбрать номер обращения, на котором требуется сравнить алгоритмы замещения.
Вторая часть является таблицей, которая отображает результаты работы методов.
В первом столбце таблице написано название метода, затем индекс выбранной методом страницы для замещения, затем число, которое показывает через сколько обращений к буферу выбранную страницу необходимо будет загрузить обратно.

\section{Вывод}

В рамках данного раздела были выбраны средства реализации программного обеспечения метода замещения страниц с использованием нейронных сетей.
В качестве языка программирования был выбран Python, а для проектирования и обучения нейронных сетей была использована библиотека PyTorch. 

Также был проведен анализ функций в СУБД PostgreSQL и проведена модификация одной из функций для созданий обучающей выборки.

Были созданы и обучены модели, которые решают свои подзадачи в рамках метода замещения страниц: кодирование запросов обращения к страницам, кодирование страниц в буфере, кодирование истории обращений, а также модель для выбора страницы для замещения на основе скрытого представления страниц в буфере и контекста, полученного из кодировщика истории.

Точность совпадений с оптимальным алгоритмом на обучающей выборке составила 63 процента, а на тестовой -- 44.2 процента.
\chapter{Исследовательский раздел}

\section{Подбор параметров сети}
Для оценки разработанного метода вводятся следующие метрики качества:
\begin{itemize}
	\item коэффициент попадания -- отношение числа обращений к страницам, которые уже загружены в буфер, к общему числу обращений;
	\item коэффициент совпадения -- отношение количества совпавших с оптимальным алгоритмов кандидатов на замещение с общим числом запросов поиска страниц для вытеснения.
\end{itemize}

Размер скрытых слоев модели подбирался экспериментально.
Графики зависимости коэффициента совпадения в зависимости от эпохи обучения для различных размеров скрытых слоев на обучающей и тестовой выборках представлены на рисунках \ref{img:test_sizes_train} и \ref{img:test_sizes_test} соответственно.
\includeimage
{test_sizes_train} % Имя файла без расширения (файл должен быть расположен в директории inc/img/)
{f} % Обтекание (без обтекания)
{H} % Положение рисунка (см. figure из пакета float)
{0.8\textwidth} % Ширина рисунка
{Точность модели для различных размеров скрытых слое на тренировочной выборке} % Подпись рисунка

\includeimage
{test_sizes_test} % Имя файла без расширения (файл должен быть расположен в директории inc/img/)
{f} % Обтекание (без обтекания)
{H} % Положение рисунка (см. figure из пакета float)
{0.8\textwidth} % Ширина рисунка
{Точность модели для различных размеров скрытых слое на тестовой выборке} % Подпись рисунка

Исходя из полученных результатов, настраиваемые параметры модели: $d_z$, $d_b$, $d_h$ и $d_v$ были выбраны равными 448, а $d_f$ -- 32.

\section{Сравнение с аналогами}
Сравнение коэффициентов попадания для разработанного метода и существующих аналогов приведено на рисунке \ref{img:hits_comp}.

\includeimage
{hits_comp} % Имя файла без расширения (файл должен быть расположен в директории inc/img/)
{f} % Обтекание (без обтекания)
{H} % Положение рисунка (см. figure из пакета float)
{0.8\textwidth} % Ширина рисунка
{Коэффициент попадания в зависимости от числа обращений для различных методов} % Подпись рисунка

Из графиков видно, что коэффициент попадания для разработанного метода в среднем на 0.02 выше чем для алгоритма clock, который в настоящее время используется в PostgreSQL.
Также коэффициент попадания для разработанного метода на 0.08 ниже, чем у оптимального алгоритма.
Таким образом, разработанный метод лучше существующий аналогов, но все еще имеет возможность для улучшения.

\section{Сравнение различных размеров буфера}

Было проведено сравнение точности модели на тренировочной и тестовой выборках при различных размерах буфера: 64, 128 и 256 страниц.
Результаты для тренировочной и тестовой выборках представлена на рисунках \ref{img:buf_sizes_train_res} и \ref{img:buf_sizes_test_res} соответственно.

\includeimage
{buf_sizes_train_res} % Имя файла без расширения (файл должен быть расположен в директории inc/img/)
{f} % Обтекание (без обтекания)
{H} % Положение рисунка (см. figure из пакета float)
{0.8\textwidth} % Ширина рисунка
{Коэффициент совпадения в зависимости от эпохи обучения для различных размеров буфера на тренировочной выборке} % Подпись рисунка

\includeimage
{buf_sizes_test_res} % Имя файла без расширения (файл должен быть расположен в директории inc/img/)
{f} % Обтекание (без обтекания)
{H} % Положение рисунка (см. figure из пакета float)
{0.8\textwidth} % Ширина рисунка
{Коэффициент совпадения в зависимости от эпохи обучения для различных размеров буфера на тестовой выборке} % Подпись рисунка

Из полученных графиков видно, что чем больше размер буфера, тем ниже точность модели как на тренировочной, так и на тестовой выборках.
Это связанно с тем, что увеличение размера буфера приводит к увеличению числа обучаемых параметров модели и числа возможных вариантов ответов.
Таким образом, метод может оказаться неэффективным при большом размере буфера.
По умолчанию разделяемых кэш буфер Postgres содержит 128 страниц.
Оценка коэффициентов попадания для разработанного метода и аналогов на таком размере буфера показала, что разработанный метод в среднем на два процента лучше аналогов по этому показателю.

\section{Вывод}
Проведенные исследования позволили определить настраиваемые параметры модели, такие как размеры скрытых слоев $d_z$, $d_b$, $d_h$, $d_v = 448$, $d_f = 32$, обеспечивающие баланс между точностью и вычислительной сложностью. 
Разработанный метод продемонстрировал улучшение коэффициента попадания на 0.02 по сравнению с алгоритмом clock, используемым в PostgreSQL, что подтверждает его практическую эффективность. 
Однако отставание на 0.08 от оптимального алгоритма указывает на потенциал для дальнейшей оптимизации.

Анализ влияния размера буфера показал, что увеличение его объема  приводит к снижению точности модели из-за роста числа обучаемых параметров и вариантов вытеснения.
Результаты, полученные для размера буфера по умолчанию, показали улучшение по сравнению с существующими аналогами по введенным метрикам качества.
Это подтверждает целесообразность внедрения метода в реальные системы с аналогичными настройками.

Таким образом, предложенный метод является перспективным решением для управления замещением страниц в разделяемом кэш буфере PostgreSQL, которое может повысить производительность системы управления базами данных за счет меньшего числа операций, взаимодействующих с диском.
\chapter*{ЗАКЛЮЧЕНИЕ}
\addcontentsline{toc}{chapter}{ЗАКЛЮЧЕНИЕ}

В ходе выполнения выпускной квалификационной работы был спроектирован и разработан метод замещения страниц в разделяемом кэш буфере PostgreSQL с использованием нейронных сетей. В ходе выполнения работы были выполнены следующие задачи:
\begin{itemize}
	\item проведено сравнение существующих методов замещения страниц;
	\item описан и спроектирован метод замещения страниц с использованием нейронных сетей;
	\item разработано программное обеспечение для предложенного метода;
	\item проведено сравнение разработанного метода с существующими аналогами по коэффициентам совпадения и попадания.
\end{itemize}

Цель работы достигнута.

\makebibliography

\begin{appendices}
	\chapter{Модуль модели}
	%TODO update file
	\includelisting
	{simple_net.py} % Имя файла с расширением (файл должен быть расположен в директории inc/lst/)
	{Модель для классификации самолетов} % Подпись листинга
	\includelisting
	{dataset_handler.py} % Имя файла с расширением (файл должен быть расположен в директории inc/lst/)
	{Класс для взаимодействия с обучаемым данными} % Подпись листинга
	\includelisting
	{INetworkController.py} % Имя файла с расширением (файл должен быть расположен в директории inc/lst/)
	{Алгоритм прохода одной эпохи} % Подпись листинга
	\includelisting
	{NetworkController.py} % Имя файла с расширением (файл должен быть расположен в директории inc/lst/)
	{Класс для взаимодействия с моделью} % Подпись листинга
	
	\chapter{Презентация}

\end{appendices}

\end{document}