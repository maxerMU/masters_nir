\chapter*{ВВЕДЕНИЕ}
\addcontentsline{toc}{chapter}{ВВЕДЕНИЕ}

Современные базы данных, такие как PostgreSQL, сталкиваются с постоянно растущими требованиями к производительности и эффективности управления ресурсами.
Одним из ключевых аспектов работы базы данных является управление памятью, включая работу с разделяемым кэш буфером \cite{yuan2022learned}.

Загрузка данных с диска занимает гораздо больше времени, чем из оперативной памяти, поэтому современные системы управления базами данных используют область в оперативной памяти в качестве буфера для кэширования недавно просмотренных страниц, чтобы в будущем запросы к страницам в буфере выполнялись быстрее. 
Обычно буфер делится на части одинакового размера, где каждая часть может содержать страницу. 
Когда транзакция базы данных запрашивает страницу, которая в данный момент не хранится в буфере, она должна быть загружена в буфер.
Если для кэширования этой страницы больше нет места, то одна из страниц в буфере должна быть вытеснена, чтобы освободить место для новой запрашиваемой страницы. 
Выбор такой страницы важен для уменьшения задержки доступа.
Если все время для вытеснения будет выбираться страница, к которой в скором времени опять произойдет обращение, то производительность может ухудшиться до случая, когда данные в основном берутся с диска.

Для выбора страницы, которую надо исключить из буфера, применяются различные эвристические алгоритмы.
Все эти алгоритмы являются приближением оптимального алгоритма и не учитывают структуру конкретной рабочей нагрузки.
Если алгоритм замещения страниц будет учитывать особенности рабочей нагрузки, то число операций чтения и записи на диск может быть снижено, что приведет к повышению производительности системы.

Помимо систем управления базами данных методы замещения страниц используются в операционных системах, аппаратном и программным кэше, а также в других местах, где присутствует два типа памяти, один из которых меньше по объему и быстрее по скорости доступа.

Целью данной работы является разработка метода замещения страниц в разделяемом кэш буфере postgres с использованием нейронных сетей.

Для достижения поставленной цели требуется выполнить следующие задачи:
\begin{itemize}
	\item сравнить существующие методы замещения страниц;
	\item описать и спроектировать метод замещения страниц с использованием нейронных сетей;
	\item разработать программное обеспечение для предложенного метода;
	\item провести сравнение разработанного метода с существующими аналогами по коэффициентам совпадения и попадания.
\end{itemize}