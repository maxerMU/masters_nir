
%\begin{essay}{}

 
%\end{essay}

\setcounter{page}{3}

\chapter*{РЕФЕРАТ}
Расчетно-пояснительная записка к научно-исследовательской работе содержит  \begin{NoHyper}\pageref{LastPage}\end{NoHyper} страниц, \totfig~иллюстраций, \tottab~таблица, 8 источников.

Научно-исследовательская работа представляет собой изучение предметной области виртуализации, описание основных методов, а также преимуществ и недостатков каждого из них. Рассмотрены различные подходы виртуализации программного обеспечения. Представлено описание методов полной, неполной, аппаратной, контейнерной и паравиртуализации. Проведено сравнение контейнеров и виртуальных машин.

Ключевые слова: виртуализация, гипервизор, виртуальная машина, контейнер, гостевая операционная система.