\setcounter{page}{5}

\chapter*{РЕФЕРАТ}
\addcontentsline{toc}{chapter}{РЕФЕРАТ}

%Расчетно-пояснительная записка к выпускной квалификационной работе содержит  \begin{NoHyper}\pageref{LastPage}\end{NoHyper} страницы, \totfig~иллюстраций, 29 источников, 2 приложения.

%В данной выпускной квалификационной работе представлена разработка метода замещения страниц в СУБД Postgres с использованием нейронных сетей.

%Данная выпускная квалификационная работа посвящена исследованию методов замещения страниц, которые могут быть использованы в разделяемом кэш буфере СУБД Postgres.
%Все методы замещения страниц являются приближением оптимального алгоритма, который предполагает исключать страницу, к которой дольше всего не будет обращений в будущем.
%В данной работе описывается метод замещения страниц, который выбирает страницу для замещения с использованием нейронных сетей.

%В работе рассмотрена задача замещения страниц в СУБД Postgres. 
%Рассмотрены различные методы замещения страниц. 
%Проведен сравнительных анализ различных архитектур нейронных сетей, а также подходов к их обучению. 
%Реализован метод замещения страниц с использование нейронных сетей и проведено исследование точности обученной модели в зависимости от различных значений настраиваемых параметров.
%Проведено сравнение коэффициентов точности и попадания разработанного метода с существующими аналогами.



Расчетно-пояснительная записка к выпускной квалификационной работе «Метод замещения страниц в разделяемом кэш буфере Postgres с использованием нейронных сетей» содержит \begin{NoHyper}\pageref{LastPage}\end{NoHyper}страниц, 4 части, \totfig~рисунков, 0 таблиц и список используемых источников из 29 наименований.

Ключевые слова: замещение страниц, рекуррентные сети, Postgres.
	
Объект разработки -- метод замещения страниц в разделяемом кэш буфере Postgres.

Цель работы: разработка метода замещения страниц в разделяемом кэш буфере Postgres с использованием нейронных сетей. 

В первой части работы выполнен анализ существующих методов замещения страниц. 
Изучены принципы работы разделяемого кэш буфера PostgreSQL.
Проведен сравнительный анализ нейронных сетей, подходящих для решения задачи.
Сформулирована цель и формализована постановка задачи в виде IDEF0-диаграммы.

Во второй части разработан метод замещения страниц в разделяемом кэш буфере PostgreSQL.
Описаны основные особенности предлагаемого метода.
Сформулированы ограничения предметной области.
Изложены основные этапы разрабатываемого метода в виде детализированной диаграммы IDEF0 и схем алгоритмов.
Выделены функции и структуры СУБД PostgreSQL, которые используются при работе с кэш буфером.

В третьей части обоснован выбор программных средств реализации метода замещения страниц в разделяемом кэш буфере.
Создана обучающая выборка с помощью логирования обращений к буферу. 
Приведены примеры работы программы. 
Описаны использованные методы тестирования программного обеспечения и приведены его результаты.

В четвертой части проведено исследование разработанного метода и выявлена зависимость коэффициентов попадания и совпадения от количества обращений к страницам на тестовой выборке.
Проведено сравнение полученных результатов и значений этих метрик для существующих аналогов.

Разработанный метод замещения страниц может быть использован в СУБД Postgres. 
Использование метода позволит повысить коэффициент попадания в разделяемом кэш буфере, что должно привести к уменьшению времени отклика системы. 