\chapter*{ВВЕДЕНИЕ}
\addcontentsline{toc}{chapter}{ВВЕДЕНИЕ}

С течением времени были разработаны два основных подхода для пре-
одоления перегрузки памяти [1]. Один из них, называемый свопингом, заклю-
чается в полном размещении процесса в памяти при его запуске на некоторое
время, а затем в его сбросе на диск. Бездействующие процессы хранятся на
диске, освобождая оперативную память. Второй подход, называемый вирту-
альной памятью, позволяет программам запускаться, даже если они находятся
в оперативной памяти лишь частично. В этом случае страницы загружаются
в оперативную память и сбрасываются на диск по мере необходимости во
время выполнения программы.
Процесс сброса страниц на диск и возвращение их в оперативную память
является трудоемким, так как требует обращения к ресурсам жесткого диска и
упирается в его пропускную способность. Минимизация числа таких операций
может повысить производительность системы [2].
Алгоритмы замещения страниц являются важной частью управления
памятью в операционных системах. Они определяют, какие страницы памяти
должны быть удалены из физической памяти при нехватке места и какие
страницы должны быть загружены обратно при необходимости. Различные
алгоритмы замещения страниц имеют разные стратегии и критерии для
принятия решений о замещении страниц.
Целью данной работы является сравнительный анализ методов замеще-
ния страниц. Для достижения поставленной цели нужно выполнить следующие
задачи:

Для достижения поставленной цели требуется выполнить следующие задачи:
\begin{itemize}
	\item провести анализ предметной области управления памятью;
	\item сформулировать критерии сравнения методов;
	\item провести сравнительный анализ методов замещения страниц.
\end{itemize}