\chapter*{ВВЕДЕНИЕ}
\addcontentsline{toc}{chapter}{ВВЕДЕНИЕ}

Многим организациям для поддержания процесса своей работы требуется несколько серверов с запущенными на них различными приложениями.
К примеру, в компании может быть почтовый сервер, файловое хранилище, веб сервис.
Каждый из этих серверов должен запускаться на определенной операционной системе.

Запуск каждого сервиса на отдельной физической машине влечет за собой большие расходы для компании по сравнению с запуском всего на одном сервере \cite{tanenbaumOS}.

Одним из вариантов решения данной проблемы является использование технологии виртуализации, которая позволяет запускать несколько виртуальных машин на одной физической.
Преимуществом такого подхода является то, что авария одной виртуальной машины не приводит к аварии на других.

Также виртуализация позволяет запускать устаревшие приложения на операционных системах, которые больше не поддерживаются на текущем физическом оборудовании.

Целью данной работы является классификация методов виртуализации программного обеспечения.

Для достижения поставленной цели требуется выполнить следующие задачи:
\begin{itemize}
	\item провести анализ предметной области виртуализации;
	\item сформулировать критерии сравнения методов;
	\item провести сравнительный анализ методов виртуализации.
\end{itemize}