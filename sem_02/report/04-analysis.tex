\chapter{Анализ предметной области}

\section{Управление памятью}

Виртуальная память представляет собой ключевую концепцию в управле-
нии памятью современных компьютерных систем. Она позволяет программам
использовать объем оперативной памяти, превышающий физически доступ-
ный, за счет автоматического перемещения данных между основной памятью
и вторичным хранилищем . Это достигается благодаря использованию вир-
туальных адресов, которые транслируются в физические адреса с помощью
аппаратных средств.

Каждая программа работает с собственным адресным пространством,
которое разбивается на страницы, представляющие собой непрерывные диа-
пазоны адресов. Эти страницы не обязательно должны все одновременно
находиться в оперативной памяти для выполнения программы, что позволяет
эффективно использовать доступную память.

Когда программа обращается к данным, которые уже находятся в фи-
зической памяти, аппаратное обеспечение обеспечивает необходимое отоб-
ражение адресов. Когда программа пытается получить доступ к странице,
которая присутствует в виртуальном адресном пространстве, но отсутствует
в физической памяти возникает системное прерывания отсутствия страницы
после чего управление передается операционной системе.

Операционная система реагирует на ошибку отсутствия страницы, выби-
рая страницу, которая редко используется, и сбрасывая её содержимое на диск,
если оно уже не находится там. Затем система извлекает нужную страницу с
диска и помещает её в освободившееся место в памяти. После этого в таблицы
вносятся соответствующие изменения, и прерванная команда выполняется
заново

Таблица страниц содержит сведения о каждой странице, включая номер
страничного блока, который является ключевым элементом страничного отоб-
ражения. Также в информации содержится бит присутствия-отсутствия, если
он равен 1, запись активна и может быть использована, иначе соответствую-
щая виртуальная страница в данный момент отсутствует в памяти, и любое
обращение к такой записи вызывает ошибку отсутствия страницы. Биты защиты указывают на тип доступа, который разрешен для страницы. В самом
простом случае это один бит, который равен 0 для чтения-записи и 1 для
только чтения. В более сложных системах могут быть использованы три бита,
каждый из которых разрешает чтение, запись или исполнение страницы. Биты
модификации и ссылки служат для отслеживания использования страницы.
Бит модификации автоматически устанавливается при записи в страницу и
помогает операционной системе определить, нужно ли сохранять страницу
на диск при ее выгрузке из памяти. Бит ссылки устанавливается при любом
обращении к странице и помогает операционной системе определить, какую
страницу следует выгрузить при возникновении ошибки отсутствия страницы.

Изучив поведение программ, разработчики компьютерных систем при-
шли к выводу, что большинство программ часто обращаются к ограниченному
набору страниц. Из-за этого только небольшая часть записей в таблице страниц активно используется, а остальная практически не задействуется. На
основе этого наблюдения для повышения производительности системы было
предложены добавить в аппаратуру специальное устройство, которое называ-
ется TLB, и отвечает за трансляцию виртуальных адресов в физические для
самых используемых страниц.

При использовании программного управления TLB существует 2 типа
ошибок: программные и аппаратные. Программная ошибка возникает, когда
страница отсутствует в TLB, но есть в памяти, и ее можно исправить простым
обновлением TLB без обращения к диску. Это занимает 10-20 машинных команд и несколько наносекунд. Аппаратная ошибка возникает, когда страница
отсутствует в памяти и требуется обращение к диску, что занимает несколько миллисекунд. Она обрабатывается значительно медленнее программной
ошибки.

При возникновении ошибки отсутствия страницы [3], операционная
система должна определить, какую страницу из памяти выселить, чтобы
освободить место для загружаемой страницы. Если страница, которую нужно
выселить, была изменена с момента загрузки в память, то ее содержимое
должно быть обновлено на диске. Если страница не подвергалась изменениям
и дисковая копия актуальна, то перезапись не требуется. В этом случае новая
страница просто замещает старую.

\section{Оптимальный алгоритм}

Оптимальной алгоритм предлагает вытеснять страницу, которая будет
без ссылок в течение самого длительного времени. Этот алгоритм может
быть реализован только во втором идентичном прогоне при условии истории
использования страниц во время первого запуска [4]. Обычно у операционной
системы этой истории нет, особенно в приложениях, получающих внешние
данные. Адрес содержание и точное время ввода могут сильно изменить
порядок и время обращения к страницам. Оптимальный алгоритм может
быть использован для оценки других алгоритмов замещения страниц, которые
могут быть применены и при первом прогоне.

\section{Алгоритм исключения недавно	использовавшейся страницы}

Для того чтобы собирать статистику использования страниц виртуальной памяти, большинство компьютеров используют два бита состояния для
каждой страницы. Бит R устанавливается при обращении к странице, а бит
M устанавливается, когда страница изменяется.

Если аппаратура не поддерживает эти биты, то они могут быть созданы
с помощью механизмов операционной системы. При запуске процесса все
записи в его таблице страниц помечаются как отсутствующие в памяти. Когда
происходит обращение к странице, возникает ошибка отсутствия страницы,
и операционная система устанавливает бит R, изменяет запись в таблице
страниц, устанавливая режим доступа только для чтения, и перезапускает
команду. Если страница впоследствии изменяется, возникает другая ошибка
страницы, позволяющая операционной системе установить бит M и изменить
режим доступа к странице на чтение-запись.

Идея алгоритма исключения недавно использовавшейся страницы заключается в следующем: при запуске процесса оба этих бита для всех страниц
устанавливаются в 0 операционной системой. При каждом прерывании от
таймера бит R сбрасывается, чтобы отличить страницы, к которым не было
обращений в последнее время, от тех, к которым были такие обращения.

При возникновении ошибки отсутствия страницы операционная систе-
ма анализирует все страницы и на основе текущих значений битов R и M
разделяет их на четыре категории:
\begin{enumerate}
	\item К которым не было ни обращений, ни модификаций в последнее время.
	\item К которым не было обращений в последнее время, но были модификации.
	\item К которым были обращения в последнее время, но не было модификаций.
	\item К которым были и обращения, и модификации в последнее время.
\end{enumerate}

Для замещения выбирается произвольная страница из самого низкого непустого класса.

\section{Алгоритм first in first out и его модификации}

Операционная система ведет список всех страниц, находящихся в памяти в данный момент. Недавно поступившие страницы находятся в конце
списка, а те, что поступили раньше всех, находятся в начале. Если возникает
ошибка отсутствия страницы, удаляется страница из начала списка, и в конец
добавляется новая страница.

Алгоритм второй шанс является простой модификацией алгоритма FIFO
и решает проблему удаления часто востребуемой страницы. Для этого используется проверка бита R самой старой страницы. Если значение этого бита
равно нулю, то это означает, что страница не только старая, но и невостребованная, поэтому она сразу же удаляется. Если бит R имеет значение 1, то
он сбрасывается, а страница помещается в конец списка страниц, а время ее
загрузки обновляется, как будто она только что поступила в память. Затем поиск продолжается.

Алгоритм часы является улучшением алгоритма второй шанс. Он основан на идее использования циклического списка страниц, представленного в
виде часов, где стрелка указывает на самую старую страницу.

Принцип работы алгоритма часы следующий:
\begin{enumerate}
	\item В начале работы алгоритма все страницы помещаются в циклический
	список в виде часов, где каждая страница имеет бит R, который указывает на ее актуальность.
	\item При возникновении ошибки отсутствия страницы проверяется страница,
	на которую указывает стрелка в циклическом списке.
	\item Если бит R этой страницы равен 0, она удаляется из памяти, на ее
	место загружается новая страница, и стрелка сдвигается вперед на одну
	позицию.
	\item Если бит R равен 1, он сбрасывается, и стрелка перемещается на следующую страницу в списке.
	\item Этот процесс повторяется до тех пор, пока не будет найдена страница с
	битом R = 0.
\end{enumerate}

\section{Алгоритм замещения наименее востребованной страницы}

Алгоритм замещения наименее востребованной страницы основан на идее, что страницы, которые долгое время не были востребованы, скорее всего
останутся невостребованными, в то время как страницы, которые интенсивно
использовались в последнее время, вероятно будут снова востребованы. Поэтому стратегия замещения страниц в этом алгоритме основана на выборе
наименее востребованной страницы для удаления.

Для реализации алгоритма NRU каждая страница в памяти связывается
с программным счетчиком, который имеет начальное значение 0. При каждом
прерывании от таймера операционная система сканирует все страницы в
памяти. Для каждой страницы к счетчику добавляется значение бита R,
который равен 0 или 1. Таким образом, счетчики позволяют приблизительно
отслеживать частоту обращений к каждой странице.

При возникновении ошибки отсутствия страницы для замещения выбирается та страница, у которой счетчик имеет наименьшее значение, то есть та
страница, которая дольше всего не была востребована.

Основная проблема этого алгоритма заключается в том, что он никогда не сбрасывает счетчики и страницы, которые активно использовались в
прошлом, и сейчас не востребованы все равное будут оставаться в памяти [5].

Для борьбы с этой проблемой существует алгоритм старения, который предлагает при каждом прерывании таймера не прибавлять 1 к счетчику, а
делать сдвиг вправо и прибавлять 1 к левому биту счетчика.

\section{Алгоритм рабочий набор}

Процессы начинают работу без каких-либо страниц в памяти,
что приводит к ошибкам отсутствия страниц при первом обращении к данным.
Операционная система загружает страницы по мере необходимости. Постепенно процесс получает большинство необходимых ему страниц и начинает работу более стабильно. Рабочий набор страниц, используемых процессом в данный
момент, важен для эффективной работы. Многие системы замещения страниц
стремятся отслеживать рабочий набор каждого процесса и обеспечивать его
присутствие в памяти, перед перезапуском процесса.

Для реализации модели рабочего набора необходимо, чтобы операционная система отслеживала, какие страницы именно входят в рабочий набор.
Имея эту информацию, можно использовать следующий алгоритм замещения
страниц: при возникновении ошибки отсутствия страницы следует выселить
ту страницу, которая не принадлежит рабочему набору.

Рабочий набор представляет собой набор страниц, используемых в k
последних обращениях к памяти. Для реализации алгоритма можно отслеживать страницы, использованные в k последних миллисекундах выполнения,
вместо поиска страниц, используемых в k последних обращениях. Для получения этой информации можно добавить специальное поле в таблицу страниц
и обновлять его на основе бита R по тику таймера.

Если "возраст" страницы превышает заранее выбранное значение на
момент возникновения ошибки, она становится кандидатом на замену. В
противном случае удаляется страница с самым большим "возрастом или случайная, если у всех страниц одинаковый параметр.

\section{Алгоритм WSClock}

Алгоритм WSClock (Working Set Clock) является модификацией алгоритма рабочего набора и базируется на структуре данных, аналогичной
циклическому списку страничных блоков, используемой в алгоритме часы.
Основные принципы работы данного алгоритма следующие:
\begin{enumerate}
	\item Создается пустой циклический список страничных блоков.
	\item При загрузке первой страницы она добавляется в список. По мере загрузки следующих страниц они также попадают в список, формируя замкнутое кольцо.
	\item В каждой записи списка содержится поле времени последнего использования из базового алгоритма рабочего набора, а также биты R и
	M.
	\item При возникновении ошибки отсутствия страницы сначала проверяется
	страница, на которую указывает "стрелка" в списке. Если бит R установлен в 1, это означает, что страница была использована в течение	текущего такта и не является идеальным кандидатом на удаление.
	\item Затем бит R устанавливается в 0, стрелка перемещается на следующую
	страницу в списке, и процесс повторяется уже для нее.
	\item После того, как бит R у страницы, на которую указывает стрелка,
	равен 0 и ее возраст превышает заданное значение, а также страница не
	изменена, происходит замещение этой страницы.
	\item Если страница была изменена, то планируется запись на диск.
\end{enumerate}

Если стрелка проходит полный круг и хотя бы одна запись на диск
запланирована, поиск может продолжаться до тех пор, пока не будет найдена неизмененная страница. В противном случае все страницы считаются частью
рабочего набора, и замещается любая страница, которая не была изменена.
Если такой страницы нет, то замещается текущая страница.

\section{Сравнение методов замещения страниц}

Оптимальный алгоритм удаляет страницу с самым отдаленным предстоящим обращением. На практике реализовать такой алгоритм невозможно, но его можно использовать в качестве оценочного критерия.

Алгоритм исключения недавно использовавшейся страницы проводит
разбиение всех страниц, основываясь на состоянии битов M и R, на 4 класса и
проводит замещение произвольной страницы наименьшего непустого класса.

Алгоритм FIFO работает по принципу очереди и удаляет самую старую страницу. Алгоритм второй шанс борется с недостатками FIFO и перед удалением страницы проверяет не используется ли она в данный момент. Алгоритм
часы является разновидностью алгоритма второй шанс, но требует меньше времени на выполнение.

Алгоритм замещения наименее востребованной страницы стремится удалять страницы, которые не были востребованы долгое время. У этого
алгоритма есть недостаток, связанный с тем, что страница, которая активно
использовалась в прошлом, не обязательно будет востребована сейчас. Для
борьбы с этим недостатком был разработан алгоритм старения.

Алгоритм рабочего набора отслеживает набор страниц, используемых за определенный промежуток времени и замещает страницу, которая не относится
к рабочему набору. Алгоритм WSClock является оптимизацией алгоритма рабочего набора.

На практике чаще всего используются алгоритм старения и WSClock.
Оба обеспечивают неплохую производительность страничной организации
памяти и могут быть эффективно реализованы, но не лишены недостатков на определенном наборе задач.

