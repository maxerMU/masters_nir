\chapter*{ЗАКЛЮЧЕНИЕ}
\addcontentsline{toc}{chapter}{ЗАКЛЮЧЕНИЕ}
Технология виртуализации позволяет запускать несколько виртуальных машин на одном физическом сервере.

В случае когда платформа, под которую разработана гостевая операционная система, отличается от плафтормы сервера, где она будет запускать, требуется использовать гипервизоры с полной виртуализацией.

Для решения задачи изоляции процессов, которые используют одно ядро операционной системы, но разные окружения, лучше всего подойдет технология контейнеризации.

В данной научно-исследовательской работе был:
\begin{itemize}
	\item проведен анализ предметной области виртуализации;
	\item сформулированы критерии сравнения методов;
	\item проведен сравнительный анализ методов виртуализации.
\end{itemize}

В рамках работы были выполнены все поставленные задачи. Цель работы была достигнута.